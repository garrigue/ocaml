\chapter{Language extensions} \label{c:extensions}
%HEVEA\cutname{extn.html}

This chapter describes language extensions and convenience features
that are implemented in OCaml, but not described in chapter \ref{c:refman}.


%HEVEA\cutdef{section}
\section{s:letrecvalues}{Recursive definitions of values}
%HEVEA\cutname{letrecvalues.html}

(Introduced in Objective Caml 1.00)

As mentioned in section~\ref{sss:expr-localdef}, the @'let' 'rec'@ binding
construct, in addition to the definition of recursive functions,
also supports a certain class of recursive definitions of
non-functional values, such as
\begin{center}
@"let" "rec" name_1 "=" "1" "::" name_2
"and" name_2 "=" "2" "::" name_1
"in" expr@
\end{center}
which binds @name_1@ to the cyclic list "1::2::1::2::"\ldots, and
@name_2@ to the cyclic list "2::1::2::1::"\ldots
Informally, the class of accepted definitions consists of those
definitions where the defined names occur only inside function
bodies or as argument to a data constructor.

More precisely, consider the expression:
\begin{center}
@"let" "rec" name_1 "=" expr_1 "and" \ldots "and" name_n "=" expr_n "in" expr@
\end{center}
It will be accepted if each one of @expr_1 \ldots expr_n@ is
statically constructive with respect to @name_1 \ldots name_n@,
is not immediately linked to any of @name_1 \ldots name_n@,
and is not an array constructor whose arguments have abstract type.

An expression @@e@@ is said to be {\em statically constructive
with respect to} the variables @name_1 \ldots name_n@ if at least
one of the following conditions is true:
\begin{itemize}
\item @@e@@ has no free occurrence of any of @name_1 \ldots name_n@
\item @@e@@ is a variable
\item @@e@@ has the form @"fun" \ldots "->" \ldots@
\item @@e@@ has the form @"function" \ldots "->" \ldots@
\item @@e@@ has the form @"lazy" "(" \ldots ")"@
\item @@e@@ has one of the following forms, where each one of
  @expr_1 \ldots expr_m@ is statically constructive with respect to
  @name_1 \ldots name_n@, and @expr_0@ is statically constructive with
  respect to @name_1 \ldots name_n, xname_1 \ldots xname_m@:
  \begin{itemize}
  \item @"let" ["rec"] xname_1 "=" expr_1 "and" \ldots
         "and" xname_m "=" expr_m "in" expr_0@
  \item @"let" "module" \ldots "in" expr_1@
  \item @constr "("expr_1"," \ldots "," expr_m")"@
  \item @"`"tag-name "("expr_1"," \ldots "," expr_m")"@
  \item @"[|" expr_1";" \ldots ";" expr_m "|]"@
  \item @"{" field_1 "=" expr_1";" \ldots ";" field_m = expr_m "}"@
  \item @"{" expr_1 "with" field_2 "=" expr_2";" \ldots ";"
             field_m = expr_m "}"@ where @expr_1@ is not immediately
             linked to @name_1 \ldots name_n@
  \item @"(" expr_1"," \ldots "," expr_m ")"@
  \item @expr_1";" \ldots ";" expr_m@
  \end{itemize}
\end{itemize}

An expression @@e@@ is said to be {\em immediately linked to} the variable
@name@ in the following cases:
\begin{itemize}
\item @@e@@ is @name@
\item @@e@@ has the form @expr_1";" \ldots ";" expr_m@ where @expr_m@
   is immediately linked to @name@
\item @@e@@ has the form @"let" ["rec"] xname_1 "=" expr_1 "and" \ldots
   "and" xname_m "=" expr_m "in" expr_0@ where @expr_0@ is immediately
   linked to @name@ or to one of the @xname_i@ such that @expr_i@
   is immediately linked to @name@.
\end{itemize}

\section{s:recursive-modules}{Recursive modules}
\ikwd{module\@\texttt{module}}
\ikwd{and\@\texttt{and}}

(Introduced in Objective Caml 3.07)

% TODO: relaxed syntax

\begin{syntax}
definition:
        ...
      | 'module' 'rec' module-name ':' module-type '=' module-expr \\
        { 'and' module-name ':' module-type '=' module-expr }
;
specification:
        ...
      | 'module' 'rec' module-name ':' module-type
                 { 'and' module-name':' module-type }
\end{syntax}

Recursive module definitions, introduced by the @"module rec"@ \ldots
@"and"@ \ldots\ construction, generalize regular module definitions
@'module' module-name '=' module-expr@ and module specifications
@'module' module-name ':' module-type@ by allowing the defining
@module-expr@ and the @module-type@ to refer recursively to the module
identifiers being defined.  A typical example of a recursive module
definition is:
\begin{camlexample}{verbatim}
\begin{caml}
\begin{camlinput}
module rec A : sig
  type t = Leaf of string | Node of ASet.t
  val compare: t -> t -> int
end = struct
  type t = Leaf of string | Node of ASet.t
  let compare t1 t2 =
    match (t1, t2) with
    | (Leaf s1, Leaf s2) -> Stdlib.compare s1 s2
    | (Leaf _, Node _) -> 1
    | (Node _, Leaf _) -> -1
    | (Node n1, Node n2) -> ASet.compare n1 n2
end
and ASet
  : Set.S with type elt = A.t
  = Set.Make(A)
\end{camlinput}
\end{caml}
\end{camlexample}
It can be given the following specification:
\begin{camlexample}{signature}
\begin{caml}
\begin{camlinput}
module rec A : sig
  type t = Leaf of string | Node of ASet.t
  val compare: t -> t -> int
end
and ASet : Set.S with type elt = A.t
\end{camlinput}
\end{caml}
\end{camlexample}

This is an experimental extension of OCaml: the class of
recursive definitions accepted, as well as its dynamic semantics are
not final and subject to change in future releases.

Currently, the compiler requires that all dependency cycles between
the recursively-defined module identifiers go through at least one
``safe'' module.  A module is ``safe'' if all value definitions that
it contains have function types @typexpr_1 '->' typexpr_2@.  Evaluation of a
recursive module definition proceeds by building initial values for
the safe modules involved, binding all (functional) values to
@'fun' '_' '->' 'raise' @"Undefined_recursive_module".  The defining
module expressions are then evaluated, and the initial values
for the safe modules are replaced by the values thus computed.  If a
function component of a safe module is applied during this computation
(which corresponds to an ill-founded recursive definition), the
"Undefined_recursive_module" exception is raised at runtime:

\begin{camlexample}{verbatim}
\begin{caml}
\begin{camlinput}
module rec M: sig val f: unit -> int end = struct let f () = N.x end
and N:sig val x: int end = struct let x = M.f () end
\end{camlinput}
\begin{camloutput}
Exception: Undefined_recursive_module ("exten.etex", 1, 43).
\end{camloutput}
\end{caml}
\end{camlexample}

If there are no safe modules along a dependency cycle, an error is raised

\begin{camlexample}{verbatim}
\begin{caml}
\begin{camlinput}
module rec M: sig <<val x: int>> end = <<struct let x = N.y end>>
and N:sig <<val x: int>> val y:int end = struct let x = M.x let y = 0 end
\end{camlinput}
\begin{camlerror}
Error: Cannot safely evaluate the definition of the following cycle
       of recursively-defined modules: M -> N -> M.
       There are no safe modules in this cycle (see manual section 8.2).
  Module M defines an unsafe value, x .
  Module N defines an unsafe value, x .
\end{camlerror}
\end{caml}
\end{camlexample}

Note that, in the @specification@ case, the @module-type@s must be
parenthesized if they use the @'with' mod-constraint@ construct.

\section{s:private-types}{Private types}
%HEVEA\cutname{privatetypes.html}
\ikwd{private\@\texttt{private}}

Private type declarations in module signatures, of the form
"type t = private ...", enable libraries to
reveal some, but not all aspects of the implementation of a type to
clients of the library.  In this respect, they strike a middle ground
between abstract type declarations, where no information is revealed
on the type implementation, and data type definitions and type
abbreviations, where all aspects of the type implementation are
publicized.  Private type declarations come in three flavors: for
variant and record types (section~\ref{ss:private-types-variant}),
for type abbreviations (section~\ref{ss:private-types-abbrev}),
and for row types (section~\ref{ss:private-rows}).

\subsection{ss:private-types-variant}{Private variant and record types}


(Introduced in Objective Caml 3.07)

\begin{syntax}
type-representation:
          ...
        | '=' 'private' [ '|' ] constr-decl { '|' constr-decl }
        | '=' 'private' record-decl
\end{syntax}

Values of a variant or record type declared @"private"@
can be de-structured normally in pattern-matching or via
the @expr '.' field@ notation for record accesses.  However, values of
these types cannot be constructed directly by constructor application
or record construction.  Moreover, assignment on a mutable field of a
private record type is not allowed.

The typical use of private types is in the export signature of a
module, to ensure that construction of values of the private type always
go through the functions provided by the module, while still allowing
pattern-matching outside the defining module.  For example:
\begin{camlexample}{verbatim}
\begin{caml}
\begin{camlinput}
module M : sig
  type t = private A | B of int
  val a : t
  val b : int -> t
end = struct
  type t = A | B of int
  let a = A
  let b n = assert (n > 0); B n
end
\end{camlinput}
\end{caml}
\end{camlexample}
Here, the @"private"@ declaration ensures that in any value of type
"M.t", the argument to the "B" constructor is always a positive integer.

With respect to the variance of their parameters, private types are
handled like abstract types. That is, if a private type has
parameters, their variance is the one explicitly given by prefixing
the parameter by a `"+"' or a `"-"', it is invariant otherwise.

\subsection{ss:private-types-abbrev}{Private type abbreviations}

(Introduced in Objective Caml 3.11)

\begin{syntax}
type-equation:
          ...
        | '=' 'private' typexpr
\end{syntax}

Unlike a regular type abbreviation, a private type abbreviation
declares a type that is distinct from its implementation type @typexpr@.
However, coercions from the type to @typexpr@ are permitted.
Moreover, the compiler ``knows'' the implementation type and can take
advantage of this knowledge to perform type-directed optimizations.

The following example uses a private type abbreviation to define a
module of nonnegative integers:
\begin{camlexample}{verbatim}
\begin{caml}
\begin{camlinput}
module N : sig
  type t = private int
  val of_int: int -> t
  val to_int: t -> int
end = struct
  type t = int
  let of_int n = assert (n >= 0); n
  let to_int n = n
end
\end{camlinput}
\end{caml}
\end{camlexample}
The type "N.t" is incompatible with "int", ensuring that nonnegative
integers and regular integers are not confused.  However, if "x" has
type "N.t", the coercion "(x :> int)" is legal and returns the
underlying integer, just like "N.to_int x".  Deep coercions are also
supported: if "l" has type "N.t list", the coercion "(l :> int list)"
returns the list of underlying integers, like "List.map N.to_int l"
but without copying the list "l".

Note that the coercion @"(" expr ":>" typexpr ")"@ is actually an abbreviated
form,
and will only work in presence of private abbreviations if neither the
type of @expr@ nor @typexpr@ contain any type variables. If they do,
you must use the full form @"(" expr ":" typexpr_1 ":>" typexpr_2 ")"@ where
@typexpr_1@ is the expected type of @expr@. Concretely, this would be "(x :
N.t :> int)" and "(l : N.t list :> int list)" for the above examples.

\subsection{ss:private-rows}{Private row types}
\ikwd{private\@\texttt{private}}

(Introduced in Objective Caml 3.09)

\begin{syntax}
type-equation:
          ...
        | '=' 'private' typexpr
\end{syntax}

Private row types are type abbreviations where part of the
structure of the type is left abstract. Concretely @typexpr@ in the
above should denote either an object type or a polymorphic variant
type, with some possibility of refinement left. If the private
declaration is used in an interface, the corresponding implementation
may either provide a ground instance, or a refined private type.
\begin{camlexample}{verbatim}
\begin{caml}
\begin{camlinput}
module M : sig type c = private < x : int; .. > val o : c end =
struct
  class c = object method x = 3 method y = 2 end
  let o = new c
end
\end{camlinput}
\end{caml}
\end{camlexample}
This declaration does more than hiding the "y" method, it also makes
the type "c" incompatible with any other closed object type, meaning
that only "o" will be of type "c". In that respect it behaves
similarly to private record types. But private row types are
more flexible with respect to incremental refinement. This feature can
be used in combination with functors.
\begin{camlexample}{verbatim}
\begin{caml}
\begin{camlinput}
module F(X : sig type c = private < x : int; .. > end) =
struct
  let get_x (o : X.c) = o#x
end
module G(X : sig type c = private < x : int; y : int; .. > end) =
struct
  include F(X)
  let get_y (o : X.c) = o#y
end
\end{camlinput}
\end{caml}
\end{camlexample}

A polymorphic variant type [t], for example
\begin{camlexample}{verbatim}
\begin{caml}
\begin{camlinput}
type t = [ `A of int | `B of bool ]
\end{camlinput}
\end{caml}
\end{camlexample}
can be refined in two ways. A definition [u] may add new field to [t],
and the declaration
\begin{camlexample}{verbatim}
\begin{caml}
\begin{camlinput}
type u = private [> t]
\end{camlinput}
\end{caml}
\end{camlexample}
will keep those new fields abstract. Construction of values of type
[u] is possible using the known variants of [t], but any
pattern-matching will require a default case to handle the potential
extra fields. Dually, a declaration [u] may restrict the fields of [t]
through abstraction: the declaration
\begin{camlexample}{verbatim}
\begin{caml}
\begin{camlinput}
type v = private [< t > `A]
\end{camlinput}
\end{caml}
\end{camlexample}
corresponds to private variant types. One cannot create a value of the
private type [v], except using the constructors that are explicitly
listed as present, "(`A n)" in this example; yet, when
patter-matching on a [v], one should assume that any of the
constructors of [t] could be present.

Similarly to abstract types, the variance of type parameters
is not inferred, and must be given explicitly.

\section{s:locally-abstract}{Locally abstract types}
\ikwd{type\@\texttt{type}}
\ikwd{fun\@\texttt{fun}}
%HEVEA\cutname{locallyabstract.html}


(Introduced in OCaml 3.12, short syntax added in 4.03)

\begin{syntax}
parameter:
       ...
     | '(' "type" {{typeconstr-name}} ')'
\end{syntax}

The expression @"fun" '(' "type" typeconstr-name ')' "->" expr@ introduces a
type constructor named @typeconstr-name@ which is considered abstract
in the scope of the sub-expression, but then replaced by a fresh type
variable.  Note that contrary to what the syntax could suggest, the
expression @"fun" '(' "type" typeconstr-name ')' "->" expr@ itself does not
suspend the evaluation of @expr@ as a regular abstraction would.  The
syntax has been chosen to fit nicely in the context of function
declarations, where it is generally used. It is possible to freely mix
regular function parameters with pseudo type parameters, as in:
\begin{camlexample}{verbatim}
\begin{caml}
\begin{camlinput}
let f = fun (type t) (foo : t list) -> $\ldots$
\end{camlinput}
\end{caml}
\end{camlexample}
and even use the alternative syntax for declaring functions:
\begin{camlexample}{verbatim}
\begin{caml}
\begin{camlinput}
let f (type t) (foo : t list) = $\ldots$
\end{camlinput}
\end{caml}
\end{camlexample}
If several locally abstract types need to be introduced, it is possible to use
the syntax
@"fun" '(' "type" typeconstr-name_1 \ldots typeconstr-name_n ')' "->" expr@
as syntactic sugar for @"fun" '(' "type" typeconstr-name_1 ')' "->" \ldots "->"
"fun" '(' "type" typeconstr-name_n ')' "->" expr@. For instance,
\begin{camlexample}{verbatim}
\begin{caml}
\begin{camlinput}
let f = fun (type t u v) -> fun (foo : (t * u * v) list) -> $\ldots$
let f' (type t u v) (foo : (t * u * v) list) = $\ldots$
\end{camlinput}
\end{caml}
\end{camlexample}

This construction is useful because the type constructors it introduces
can be used in places where a type variable is not allowed. For
instance, one can use it to define an exception in a local module
within a polymorphic function.
\begin{camlexample}{verbatim}
\begin{caml}
\begin{camlinput}
let f (type t) () =
  let module M = struct exception E of t end in
  (fun x -> M.E x), (function M.E x -> Some x | _ -> None)
\end{camlinput}
\end{caml}
\end{camlexample}

Here is another example:
\begin{camlexample}{verbatim}
\begin{caml}
\begin{camlinput}
let sort_uniq (type s) (cmp : s -> s -> int) =
  let module S = Set.Make(struct type t = s let compare = cmp end) in
  fun l ->
    S.elements (List.fold_right S.add l S.empty)
\end{camlinput}
\end{caml}
\end{camlexample}

It is also extremely useful for first-class modules (see
section~\ref{s:first-class-modules}) and generalized algebraic datatypes
(GADTs: see section~\ref{s:gadts}).

\lparagraph{p:polymorpic-locally-abstract}{Polymorphic syntax} (Introduced in OCaml 4.00)

\begin{syntax}
let-binding:
       ...
     | value-name ':' 'type' {{ typeconstr-name }} '.' typexpr '=' expr
;
class-field:
          ...
        | 'method' ['private'] method-name ':' 'type'
          {{ typeconstr-name }} '.' typexpr '=' expr
        | 'method!' ['private'] method-name ':' 'type'
          {{ typeconstr-name }} '.' typexpr '=' expr
\end{syntax}

The @"(type" typeconstr-name")"@ syntax construction by itself does not make
polymorphic the type variable it introduces, but it can be combined
with explicit polymorphic annotations where needed.
The above rule is provided as syntactic sugar to make this easier:
\begin{camlexample}{verbatim}
\begin{caml}
\begin{camlinput}
let rec f : type t1 t2. t1 * t2 list -> t1 = $\ldots$
\end{camlinput}
\end{caml}
\end{camlexample}
\noindent
is automatically expanded into
\begin{camlexample}{verbatim}
\begin{caml}
\begin{camlinput}
let rec f : 't1 't2. 't1 * 't2 list -> 't1 =
  fun (type t1) (type t2) -> ( $\ldots$ : t1 * t2 list -> t1)
\end{camlinput}
\end{caml}
\end{camlexample}
This syntax can be very useful when defining recursive functions involving
GADTs, see the section~\ref{s:gadts} for a more detailed explanation.

The same feature is provided for method definitions.

\section{s:first-class-modules}{First-class modules}
\ikwd{module\@\texttt{module}}
\ikwd{val\@\texttt{val}}
\ikwd{with\@\texttt{with}}
\ikwd{and\@\texttt{and}}
%HEVEA\cutname{firstclassmodules.html}


(Introduced in OCaml 3.12; pattern syntax and package type inference
introduced in 4.00; structural comparison of package types introduced in 4.02.;
fewer parens required starting from 4.05)

\begin{syntax}
typexpr:
      ...
    | '(''module' package-type')'
;
module-expr:
      ...
    | '(''val' expr [':' package-type]')'
;
expr:
      ...
    | '(''module' module-expr [':' package-type]')'
;
pattern:
      ...
    | '(''module' module-name [':' package-type]')'
;
package-type:
      modtype-path
    | modtype-path 'with' package-constraint { 'and' package-constraint }
;
package-constraint:
          'type' typeconstr '=' typexpr
;
\end{syntax}

Modules are typically thought of as static components. This extension
makes it possible to pack a module as a first-class value, which can
later be dynamically unpacked into a module.

The expression @'(' 'module' module-expr ':' package-type ')'@ converts the
module (structure or functor) denoted by module expression @module-expr@
to a value of the core language that encapsulates this module.  The
type of this core language value is @'(' 'module' package-type ')'@.
The @package-type@ annotation can be omitted if it can be inferred
from the context.

Conversely, the module expression @'(' 'val' expr ':' package-type ')'@
evaluates the core language expression @expr@ to a value, which must
have type @'module' package-type@, and extracts the module that was
encapsulated in this value. Again @package-type@ can be omitted if the
type of @expr@ is known.
If the module expression is already parenthesized, like the arguments
of functors are, no additional parens are needed: "Map.Make(val key)".

The pattern @'(' 'module' module-name ':' package-type ')'@ matches a
package with type @package-type@ and binds it to @module-name@.
It is not allowed in toplevel let bindings.
Again @package-type@ can be omitted if it can be inferred from the
enclosing pattern.

The @package-type@ syntactic class appearing in the  @'(' 'module'
package-type ')'@ type expression and in the annotated forms represents a
subset of module types.
This subset consists of named module types with optional constraints
of a limited form: only non-parametrized types can be specified.

For type-checking purposes (and starting from OCaml 4.02), package types
are compared using the structural comparison of module types.

In general, the module expression @'(' "val" expr ":" package-type
')'@ cannot be used in the body of a functor, because this could cause
unsoundness in conjunction with applicative functors.
Since OCaml 4.02, this is relaxed in two ways:
if @package-type@ does not contain nominal type declarations ({\em
  i.e.} types that are created with a proper identity), then this
expression can be used anywhere, and even if it contains such types
it can be used inside the body of a generative
functor, described in section~\ref{s:generative-functors}.
It can also be used anywhere in the context of a local module binding
@'let' 'module' module-name '=' '(' "val" expr_1 ":" package-type ')'
 "in" expr_2@.

\lparagraph{p:fst-mod-example}{Basic example} A typical use of first-class modules is to
select at run-time among several implementations of a signature.
Each implementation is a structure that we can encapsulate as a
first-class module, then store in a data structure such as a hash
table:
\begin{camlexample}{verbatim}
\begin{caml}
\begin{camlinput}
type picture = $\ldots$
module type DEVICE = sig
  val draw : picture -> unit
  $\ldots$
end
let devices : (string, (module DEVICE)) Hashtbl.t = Hashtbl.create 17

module SVG = struct $\ldots$ end
let _ = Hashtbl.add devices "SVG" (module SVG : DEVICE)

module PDF = struct $\ldots$ end
let _ = Hashtbl.add devices "PDF" (module PDF : DEVICE)
\end{camlinput}
\end{caml}
\end{camlexample}

We can then select one implementation based on command-line
arguments, for instance:
\begin{camlexample}{verbatim}
\begin{caml}
\begin{camlinput}
let parse_cmdline () = $\ldots$
module Device =
  (val (let device_name = parse_cmdline () in
        try Hashtbl.find devices device_name
        with Not_found ->
          Printf.eprintf "Unknown device %s\n" device_name;
          exit 2)
   : DEVICE)
\end{camlinput}
\end{caml}
\end{camlexample}
Alternatively, the selection can be performed within a function:
\begin{camlexample}{verbatim}
\begin{caml}
\begin{camlinput}
let draw_using_device device_name picture =
  let module Device =
    (val (Hashtbl.find devices device_name) : DEVICE)
  in
  Device.draw picture
\end{camlinput}
\end{caml}
\end{camlexample}

\lparagraph{p:fst-mod-advexamples}{Advanced examples}
With first-class modules, it is possible to parametrize some code over the
implementation of a module without using a functor.

\begin{camlexample}{verbatim}
\begin{caml}
\begin{camlinput}
let sort (type s) (module Set : Set.S with type elt = s) l =
  Set.elements (List.fold_right Set.add l Set.empty)
\end{camlinput}
\begin{camloutput}
val sort : (module Set.S with type elt = 's) -> 's list -> 's list = <fun>
\end{camloutput}
\end{caml}
\end{camlexample}

To use this function, one can wrap the "Set.Make" functor:

\begin{camlexample}{verbatim}
\begin{caml}
\begin{camlinput}
let make_set (type s) cmp =
  let module S = Set.Make(struct
    type t = s
    let compare = cmp
  end) in
  (module S : Set.S with type elt = s)
\end{camlinput}
\begin{camloutput}
val make_set : ('s -> 's -> int) -> (module Set.S with type elt = 's) = <fun>
\end{camloutput}
\end{caml}
\end{camlexample}

\iffalse
Another advanced use of first-class module is to encode existential
types. In particular, they can be used to simulate generalized
algebraic data types (GADT). To demonstrate this, we first define a type
of witnesses for type equalities:

\begin{camlexample}{verbatim}
\begin{caml}
\begin{camlinput}
module TypEq : sig
  type ('a, 'b) t
  val apply: ('a, 'b) t -> 'a -> 'b
  val refl: ('a, 'a) t
  val sym: ('a, 'b) t -> ('b, 'a) t
end = struct
  type ('a, 'b) t = ('a -> 'b) * ('b -> 'a)
  let refl = (fun x -> x), (fun x -> x)
  let apply (f, _) x = f x
  let sym (f, g) = (g, f)
end
\end{camlinput}
\end{caml}
\end{camlexample}

We can then define a parametrized algebraic data type whose
constructors provide some information about the type parameter:

\begin{camlexample}{verbatim}
\begin{caml}
\begin{camlinput}
module rec Typ : sig
  module type PAIR = sig
    type t and t1 and t2
    val eq: (t, t1 * t2) TypEq.t
    val t1: t1 Typ.typ
    val t2: t2 Typ.typ
  end

  type 'a typ =
    | Int of ('a, int) TypEq.t
    | String of ('a, string) TypEq.t
    | Pair of (module PAIR with type t = 'a)
end = Typ
\end{camlinput}
\end{caml}
\end{camlexample}

Values of type "'a typ" are supposed to be runtime representations for
the type "'a". The constructors "Int" and "String" are easy: they
directly give a witness of type equality between the parameter "'a"
and the ground types "int" (resp. "string"). The constructor "Pair" is
more complex. One wants to give a witness of type equality between
"'a" and a type of the form "t1 * t2" together with the representations
for "t1" and "t2". However, these two types are unknown. The code above
shows how to use first-class modules to simulate existentials.

Here is how to construct values of type "'a typ":

\begin{camlexample}{verbatim}
\begin{caml}
\begin{camlinput}
let int = Typ.Int TypEq.refl

let str = Typ.String TypEq.refl

let pair (type s1) (type s2) t1 t2 =
  let module P = struct
    type t = s1 * s2
    type t1 = s1
    type t2 = s2
    let eq = TypEq.refl
    let t1 = t1
    let t2 = t2
  end in
  let pair = (module P : Typ.PAIR with type t = s1 * s2) in
  Typ.Pair pair
\end{camlinput}
\end{caml}
\end{camlexample}

And finally, here is an example of a polymorphic function that takes the
runtime representation of some type "'a" and a value of the same type,
then pretty-prints the value into a string:

\begin{camlexample}{verbatim}
\begin{caml}
\begin{camlinput}
open Typ
let rec to_string: 'a. 'a Typ.typ -> 'a -> string =
  fun (type s) t x ->
    match t with
    | Int eq -> Int.to_string (TypEq.apply eq x)
    | String eq -> Printf.sprintf "%S" (TypEq.apply eq x)
    | Pair p ->
        let module P = (val p : PAIR with type t = s) in
        let (x1, x2) = TypEq.apply P.eq x in
        Printf.sprintf "(%s,%s)" (to_string P.t1 x1) (to_string P.t2 x2)
\end{camlinput}
\end{caml}
\end{camlexample}

Note that this function uses an explicit polymorphic annotation to obtain
polymorphic recursion.
\fi

\section{s:module-type-of}{Recovering the type of a module}
%HEVEA\cutname{moduletypeof.html}

\ikwd{module\@\texttt{module}}
\ikwd{type\@\texttt{type}}
\ikwd{of\@\texttt{of}}
\ikwd{include\@\texttt{include}}

(Introduced in OCaml 3.12)

\begin{syntax}
module-type:
     ...
   | 'module' 'type' 'of' module-expr
\end{syntax}

The construction @'module' 'type' 'of' module-expr@ expands to the module type
(signature or functor type) inferred for the module expression @module-expr@.
To make this module type reusable in many situations, it is
intentionally not strengthened: abstract types and datatypes are not
explicitly related with the types of the original module.
For the same reason, module aliases in the inferred type are expanded.

A typical use, in conjunction with the signature-level @'include'@
construct, is to extend the signature of an existing structure.
In that case, one wants to keep the types equal to types in the
original module. This can done using the following idiom.
\begin{camlexample}{verbatim}
\begin{caml}
\begin{camlinput}
module type MYHASH = sig
  include module type of struct include Hashtbl end
  val replace: ('a, 'b) t -> 'a -> 'b -> unit
end
\end{camlinput}
\end{caml}
\end{camlexample}
The signature "MYHASH" then contains all the fields of the signature
of the module "Hashtbl" (with strengthened type definitions), plus the
new field "replace".  An implementation of this signature can be
obtained easily by using the @'include'@ construct again, but this
time at the structure level:
\begin{camlexample}{verbatim}
\begin{caml}
\begin{camlinput}
module MyHash : MYHASH = struct
  include Hashtbl
  let replace t k v = remove t k; add t k v
end
\end{camlinput}
\end{caml}
\end{camlexample}

Another application where the absence of strengthening comes handy, is
to provide an alternative implementation for an existing module.
\begin{camlexample}{verbatim}
\begin{caml}
\begin{camlinput}
module MySet : module type of Set = struct
  $\ldots$
end
\end{camlinput}
\end{caml}
\end{camlexample}
This idiom guarantees that "Myset" is compatible with Set, but allows
it to represent sets internally in a different way.

\section{s:signature-substitution}{Substituting inside a signature}
\ikwd{with\@\texttt{with}}
\ikwd{module\@\texttt{module}}
\ikwd{type\@\texttt{type}}
%HEVEA\cutname{signaturesubstitution.html}


\subsection{ss:destructive-substitution}{Destructive substitutions}

(Introduced in OCaml 3.12, generalized in 4.06)

\begin{syntax}
mod-constraint:
          ...
        | 'type' [type-params] typeconstr-name ':=' typexpr
        | 'module' module-path ':=' extended-module-path
\end{syntax}

A ``destructive'' substitution (@'with' ... ':=' ...@) behaves essentially like
normal signature constraints (@'with' ... '=' ...@), but it additionally removes
the redefined type or module from the signature.

Prior to OCaml 4.06, there were a number of restrictions: one could only remove
types and modules at the outermost level (not inside submodules), and in the
case of @'with type'@ the definition had to be another type constructor with the
same type parameters.

A natural application of destructive substitution is merging two
signatures sharing a type name.
\begin{camlexample}{verbatim}
\begin{caml}
\begin{camlinput}
module type Printable = sig
  type t
  val print : Format.formatter -> t -> unit
end
module type Comparable = sig
  type t
  val compare : t -> t -> int
end
module type PrintableComparable = sig
  include Printable
  include Comparable with type t := t
end
\end{camlinput}
\end{caml}
\end{camlexample}

One can also use this to completely remove a field:
\begin{camlexample}{verbatim}
\begin{caml}
\begin{camlinput}
module type S = Comparable with type t := int
\end{camlinput}
\begin{camloutput}
module type S = sig val compare : int -> int -> int end
\end{camloutput}
\end{caml}
\end{camlexample}
or to rename one:
\begin{camlexample}{verbatim}
\begin{caml}
\begin{camlinput}
module type S = sig
  type u
  include Comparable with type t := u
end
\end{camlinput}
\begin{camloutput}
module type S = sig type u val compare : u -> u -> int end
\end{camloutput}
\end{caml}
\end{camlexample}

Note that you can also remove manifest types, by substituting with the
same type.
\begin{camlexample}{verbatim}
\begin{caml}
\begin{camlinput}
module type ComparableInt = Comparable with type t = int ;;
\end{camlinput}
\begin{camloutput}
module type ComparableInt = sig type t = int val compare : t -> t -> int end
\end{camloutput}
\end{caml}
\begin{caml}
\begin{camlinput}
module type CompareInt = ComparableInt with type t := int
\end{camlinput}
\begin{camloutput}
module type CompareInt = sig val compare : int -> int -> int end
\end{camloutput}
\end{caml}
\end{camlexample}

\subsection{ss:local-substitution}{Local substitution declarations}

(Introduced in OCaml 4.08)

\begin{syntax}
specification:
          ...
        | 'type' type-subst { 'and' type-subst }
        | 'module' module-name ':=' extended-module-path
;

type-subst:
          [type-params] typeconstr-name ':=' typexpr { type-constraint }
\end{syntax}


Local substitutions behave like destructive substitutions (@'with' ... ':=' ...@)
but instead of being applied to a whole signature after the fact, they are
introduced during the specification of the signature, and will apply to all the
items that follow.

This provides a convenient way to introduce local names for types and modules
when defining a signature:

\begin{camlexample}{verbatim}
\begin{caml}
\begin{camlinput}
module type S = sig
  type t
  module Sub : sig
    type outer := t
    type t
    val to_outer : t -> outer
  end
end
\end{camlinput}
\begin{camloutput}
module type S =
  sig type t module Sub : sig type t val to_outer : t/1 -> t/2 end end
\end{camloutput}
\end{caml}
\end{camlexample}

Note that, unlike type declarations, type substitution declarations are not
recursive, so substitutions like the following are rejected:

\begin{camlexample}{toplevel}
\begin{caml}
\begin{camlinput}
$\?$ module type S = sig
    type 'a poly_list := [ `Cons of 'a * 'a <<poly_list>> | `Nil ]
  end ;;
\end{camlinput}
\begin{camlerror}
Error: Unbound type constructor poly_list
\end{camlerror}
\end{caml}
\end{camlexample}

\section{s:module-alias}{Type-level module aliases}
\ikwd{module\@\texttt{module}}
%HEVEA\cutname{modulealias.html}

(Introduced in OCaml 4.02)

\begin{syntax}
specification:
          ...
        | 'module' module-name '=' module-path
\end{syntax}

The above specification, inside a signature, only matches a module
definition equal to @module-path@. Conversely, a type-level module
alias can be matched by itself, or by any supertype of the type of the
module it references.

There are several restrictions on @module-path@:
\begin{enumerate}
\item it should be of the form \(M_0.M_1...M_n\) ({\em i.e.} without
  functor applications);
\item inside the body of a  functor, \(M_0\) should not be one of the
  functor parameters;
\item inside a recursive module definition, \(M_0\) should not be one of
  the recursively defined modules.
\end{enumerate}

Such specifications are also inferred. Namely, when @P@ is a path
satisfying the above constraints,
\begin{camlexample}{verbatim}
\begin{caml}
\begin{camlinput}
module N = P
\end{camlinput}
\end{caml}
\end{camlexample}
has type
\begin{camlexample}{signature}
\begin{caml}
\begin{camlinput}
module N = P
\end{camlinput}
\end{caml}
\end{camlexample}

Type-level module aliases are used when checking module path
equalities. That is, in a context where module name @N@ is known to be
an alias for @P@, not only these two module paths check as equal, but
@F(N)@ and @F(P)@ are also recognized as equal. In the default
compilation mode, this is the only difference with the previous
approach of module aliases having just the same module type as the
module they reference.

When the compiler flag @'-no-alias-deps'@ is enabled, type-level
module aliases are also exploited to avoid introducing dependencies
between compilation units. Namely, a module alias referring to a
module inside another compilation unit does not introduce a link-time
dependency on that compilation unit, as long as it is not
dereferenced; it still introduces a compile-time dependency if the
interface needs to be read, {\em i.e.}  if the module is a submodule
of the compilation unit, or if some type components are referred to.
Additionally, accessing a module alias introduces a link-time
dependency on the compilation unit containing the module referenced by
the alias, rather than the compilation unit containing the alias.
Note that these differences in link-time behavior may be incompatible
with the previous behavior, as some compilation units might not be
extracted from libraries, and their side-effects ignored.

These weakened dependencies make possible to use module aliases in
place of the @'-pack'@ mechanism. Suppose that you have a library
@'Mylib'@ composed of modules @'A'@ and @'B'@. Using @'-pack'@, one
would issue the command line
\begin{verbatim}
ocamlc -pack a.cmo b.cmo -o mylib.cmo
\end{verbatim}
and as a result obtain a @'Mylib'@ compilation unit, containing
physically @'A'@ and @'B'@ as submodules, and with no dependencies on
their respective compilation units.
Here is a concrete example of a possible alternative approach:
\begin{enumerate}
\item Rename the files containing @'A'@ and @'B'@ to @'Mylib__A'@ and
  @'Mylib__B'@.
\item Create a packing interface @'Mylib.ml'@, containing the
  following lines.
\begin{verbatim}
module A = Mylib__A
module B = Mylib__B
\end{verbatim}
\item Compile @'Mylib.ml'@ using @'-no-alias-deps'@, and the other
  files using @'-no-alias-deps'@ and @'-open' 'Mylib'@ (the last one is
  equivalent to adding the line @'open!' 'Mylib'@ at the top of each
  file).
\begin{verbatim}
ocamlc -c -no-alias-deps Mylib.ml
ocamlc -c -no-alias-deps -open Mylib Mylib__*.mli Mylib__*.ml
\end{verbatim}
\item Finally, create a library containing all the compilation units,
  and export all the compiled interfaces.
\begin{verbatim}
ocamlc -a Mylib*.cmo -o Mylib.cma
\end{verbatim}
\end{enumerate}
This approach lets you access @'A'@ and @'B'@ directly inside the
library, and as @'Mylib.A'@ and @'Mylib.B'@ from outside.
It also has the advantage that @'Mylib'@ is no longer monolithic: if
you use @'Mylib.A'@, only @'Mylib__A'@ will be linked in, not
@'Mylib__B'@.
%Note that in the above @'Mylib.cmo'@ is actually empty, and one could
%name the interface @'Mylib.mli'@, but this would require that all
%clients are compiled with the @'-no-alias-deps'@ flag.

Note the use of double underscores in @'Mylib__A'@ and
@'Mylib__B'@. These were chosen on purpose; the compiler uses the
following heuristic when printing paths: given a path @'Lib__fooBar'@,
if @'Lib.FooBar'@ exists and is an alias for @'Lib__fooBar'@, then the
compiler will always display @'Lib.FooBar'@ instead of
@'Lib__fooBar'@. This way the long @'Mylib__'@ names stay hidden and
all the user sees is the nicer dot names. This is how the OCaml
standard library is compiled.

\section{s:explicit-overriding-open}{Overriding in open statements}
\ikwd{open.\@\texttt{open\char33}}
%HEVEA\cutname{overridingopen.html}

(Introduced in OCaml 4.01)

\begin{syntax}
definition:
      ...
   |  'open!' module-path
;
specification:
      ...
   |  'open!' module-path
;
expr:
       ...
     | 'let' 'open!' module-path 'in' expr
;
class-body-type:
       ...
   |  'let' 'open!' module-path 'in' class-body-type
;
class-expr:
       ...
   |  'let' 'open!' module-path 'in' class-expr
;
\end{syntax}

Since OCaml 4.01, @"open"@ statements shadowing an existing identifier
(which is later used) trigger the warning 44.  Adding a @"!"@
character after the @"open"@ keyword indicates that such a shadowing is
intentional and should not trigger the warning.

This is also available (since OCaml 4.06) for local opens in class
expressions and class type expressions.

\section{s:gadts}{Generalized algebraic datatypes} \ikwd{type\@\texttt{type}}
\ikwd{match\@\texttt{match}}
%HEVEA\cutname{gadts.html}


(Introduced in OCaml 4.00)

\begin{syntax}
constr-decl:
          ...
        | constr-name ':' [ constr-args '->' ] typexpr
;
type-param:
          ...
        | [variance] '_'
\end{syntax}

Generalized algebraic datatypes, or GADTs, extend usual sum types in
two ways: constraints on type parameters may change depending on the
value constructor, and some type variables may be existentially
quantified.
Adding constraints is done by giving an explicit return type
(the rightmost @typexpr@ in the above syntax), where type parameters
are instantiated.
This return type must use the same type constructor as the type being
defined, and have the same number of parameters.
Variables are made existential when they appear inside a constructor's
argument, but not in its return type.

Since the use of a return type often eliminates the need to name type
parameters in the left-hand side of a type definition, one can replace
them with anonymous types @"_"@ in that case.

The constraints associated to each constructor can be recovered
through pattern-matching.
Namely, if the type of the scrutinee of a pattern-matching contains
a locally abstract type, this type can be refined according to the
constructor used.
These extra constraints are only valid inside the corresponding branch
of the pattern-matching.
If a constructor has some existential variables, fresh locally
abstract types are generated, and they must not escape the
scope of this branch.

\lparagraph{p:gadts-recfun}{Recursive functions}

Here is a concrete example:
\begin{camlexample}{verbatim}
\begin{caml}
\begin{camlinput}
type _ term =
  | Int : int -> int term
  | Add : (int -> int -> int) term
  | App : ('b -> 'a) term * 'b term -> 'a term

let rec eval : type a. a term -> a = function
  | Int n    -> n                 (* a = int *)
  | Add      -> (fun x y -> x+y)  (* a = int -> int -> int *)
  | App(f,x) -> (eval f) (eval x)
          (* eval called at types (b->a) and b for fresh b *)
\end{camlinput}
\end{caml}
\end{camlexample}
\begin{camlexample}{verbatim}
\begin{caml}
\begin{camlinput}
let two = eval (App (App (Add, Int 1), Int 1))
\end{camlinput}
\begin{camloutput}
val two : int = 2
\end{camloutput}
\end{caml}
\end{camlexample}
It is important to remark that the function "eval" is using the
polymorphic syntax for locally abstract types. When defining a recursive
function that manipulates a GADT, explicit polymorphic recursion should
generally be used. For instance, the following definition fails with a
type error:
\begin{camlexample}{verbatim}
\begin{caml}
\begin{camlinput}
let rec eval (type a) : a term -> a = function
  | Int n    -> n
  | Add      -> (fun x y -> x+y)
  | App(f,x) -> (eval <<f>>) (eval x)
\end{camlinput}
\begin{camlerror}
Error: This expression has type ($\textdollar$App_'b -> a) term
       but an expression was expected of type 'a
       The type constructor $\textdollar$App_'b would escape its scope
\end{camlerror}
\end{caml}
\end{camlexample}
In absence of an explicit polymorphic annotation, a monomorphic type
is inferred for the recursive function. If a recursive call occurs
inside the function definition at a type that involves an existential
GADT type variable, this variable flows to the type of the recursive
function, and thus escapes its scope. In the above example, this happens
in the branch "App(f,x)" when "eval" is called with "f" as an argument.
In this branch, the type of "f" is "($App_ 'b-> a)". The prefix "$" in
"$App_ 'b" denotes an existential type named by the compiler
(see~\ref{p:existential-names}). Since the type of "eval" is
"'a term -> 'a", the call "eval f" makes the existential type "$App_'b"
flow to the type variable "'a" and escape its scope. This triggers the
above error.

\lparagraph{p:gadts-type-inference}{Type inference}

Type inference for GADTs is notoriously hard.
This is due to the fact some types may become ambiguous when escaping
from a branch.
For instance, in the "Int" case above, "n" could have either type "int"
or "a", and they are not equivalent outside of that branch.
As a first approximation, type inference will always work if a
pattern-matching is annotated with types containing no free type
variables (both on the scrutinee and the return type).
This is the case in the above example, thanks to the type annotation
containing only locally abstract types.

In practice, type inference is a bit more clever than that: type
annotations do not need to be immediately on the pattern-matching, and
the types do not have to be always closed.
As a result, it is usually enough to only annotate functions, as in
the example above. Type annotations are
propagated in two ways: for the scrutinee, they follow the flow of
type inference, in a way similar to polymorphic methods; for the
return type, they follow the structure of the program, they are split
on functions, propagated to all branches of a pattern matching,
and go through tuples, records, and sum types.
Moreover, the notion of ambiguity used is stronger: a type is only
seen as ambiguous if it was mixed with incompatible types (equated by
constraints), without type annotations between them.
For instance, the following program types correctly.
\begin{camlexample}{verbatim}
\begin{caml}
\begin{camlinput}
let rec sum : type a. a term -> _ = fun x ->
  let y =
    match x with
    | Int n -> n
    | Add   -> 0
    | App(f,x) -> sum f + sum x
  in y + 1
\end{camlinput}
\begin{camloutput}
val sum : 'a term -> int = <fun>
\end{camloutput}
\end{caml}
\end{camlexample}
Here the return type "int" is never mixed with "a", so it is seen as
non-ambiguous, and can be inferred.
When using such partial type annotations we strongly suggest
specifying the "-principal" mode, to check that inference is
principal.

The exhaustiveness check is aware of GADT constraints, and can
automatically infer that some cases cannot happen.
For instance, the following pattern matching is correctly seen as
exhaustive (the "Add" case cannot happen).
\begin{camlexample}{verbatim}
\begin{caml}
\begin{camlinput}
let get_int : int term -> int = function
  | Int n    -> n
  | App(_,_) -> 0
\end{camlinput}
\end{caml}
\end{camlexample}


\lparagraph{p:gadt-refutation-cases}{Refutation cases} (Introduced in OCaml 4.03)

Usually, the exhaustiveness check only tries to check whether the
cases omitted from the pattern matching are typable or not.
However, you can force it to try harder by adding {\em refutation cases}:
\begin{syntax}
matching-case:
     pattern ['when' expr] '->' expr
   | pattern '->' '.'
\end{syntax}
In presence of a refutation case, the exhaustiveness check will first
compute the intersection of the pattern with the complement of the
cases preceding it. It then checks whether the resulting patterns can
really match any concrete values by trying to type-check them.
Wild cards in the generated patterns are handled in a special way: if
their type is a variant type with only GADT constructors, then the
pattern is split into the different constructors, in order to check whether
any of them is possible (this splitting is not done for arguments of these
constructors, to avoid non-termination). We also split tuples and
variant types with only one case, since they may contain GADTs inside.
For instance, the following code is deemed exhaustive:

\begin{camlexample}{verbatim}
\begin{caml}
\begin{camlinput}
type _ t =
  | Int : int t
  | Bool : bool t

let deep : (char t * int) option -> char = function
  | None -> 'c'
  | _ -> .
\end{camlinput}
\end{caml}
\end{camlexample}

Namely, the inferred remaining case is "Some _", which is split into
"Some (Int, _)" and "Some (Bool, _)", which are both untypable because
"deep" expects a non-existing "char t" as the first element of the tuple.
Note that the refutation case could be omitted here, because it is
automatically added when there is only one case in the pattern
matching.

Another addition is that the redundancy check is now aware of GADTs: a
case will be detected as redundant if it could be replaced by a
refutation case using the same pattern.

\lparagraph{p:gadts-advexamples}{Advanced examples}
The "term" type we have defined above is an {\em indexed} type, where
a type parameter reflects a property of the value contents.
Another use of GADTs is {\em singleton} types, where a GADT value
represents exactly one type. This value can be used as runtime
representation for this type, and a function receiving it can have a
polytypic behavior.

Here is an example of a polymorphic function that takes the
runtime representation of some type "t" and a value of the same type,
then pretty-prints the value as a string:
\begin{camlexample}{verbatim}
\begin{caml}
\begin{camlinput}
type _ typ =
  | Int : int typ
  | String : string typ
  | Pair : 'a typ * 'b typ -> ('a * 'b) typ

let rec to_string: type t. t typ -> t -> string =
  fun t x ->
  match t with
  | Int -> Int.to_string x
  | String -> Printf.sprintf "%S" x
  | Pair(t1,t2) ->
      let (x1, x2) = x in
      Printf.sprintf "(%s,%s)" (to_string t1 x1) (to_string t2 x2)
\end{camlinput}
\end{caml}
\end{camlexample}

Another frequent application of GADTs is equality witnesses.
\begin{camlexample}{verbatim}
\begin{caml}
\begin{camlinput}
type (_,_) eq = Eq : ('a,'a) eq

let cast : type a b. (a,b) eq -> a -> b = fun Eq x -> x
\end{camlinput}
\end{caml}
\end{camlexample}
Here type "eq" has only one constructor, and by matching on it one
adds a local constraint allowing the conversion between "a" and "b".
By building such equality witnesses, one can make equal types which
are syntactically different.

Here is an example using both singleton types and equality witnesses
to implement dynamic types.
\begin{camlexample}{verbatim}
\begin{caml}
\begin{camlinput}
let rec eq_type : type a b. a typ -> b typ -> (a,b) eq option =
  fun a b ->
  match a, b with
  | Int, Int -> Some Eq
  | String, String -> Some Eq
  | Pair(a1,a2), Pair(b1,b2) ->
      begin match eq_type a1 b1, eq_type a2 b2 with
      | Some Eq, Some Eq -> Some Eq
      | _ -> None
      end
  | _ -> None

type dyn = Dyn : 'a typ * 'a -> dyn

let get_dyn : type a. a typ -> dyn -> a option =
  fun a (Dyn(b,x)) ->
  match eq_type a b with
  | None -> None
  | Some Eq -> Some x
\end{camlinput}
\end{caml}
\end{camlexample}

\lparagraph{p:existential-names}{Existential type names in error messages}%
(Updated in OCaml 4.03.0)

The typing of pattern matching in presence of GADT can generate many
existential types. When necessary, error messages refer to these
existential types using compiler-generated names. Currently, the
compiler generates these names according to the following nomenclature:
\begin{itemize}
\item First, types whose name starts with a "$" are existentials.
\item "$Constr_'a" denotes an existential type introduced for the type
variable "'a" of the GADT constructor "Constr":
\begin{camlexample}{verbatim}
\begin{caml}
\begin{camlinput}
type any = Any : 'name -> any
let escape (Any x) = <<x>>
\end{camlinput}
\begin{camlerror}
Error: This expression has type $\textdollar$Any_'name
       but an expression was expected of type 'a
       The type constructor $\textdollar$Any_'name would escape its scope
\end{camlerror}
\end{caml}
\end{camlexample}
\item "$Constr" denotes an existential type introduced for an anonymous %$
type variable in the GADT constructor "Constr":
\begin{camlexample}{verbatim}
\begin{caml}
\begin{camlinput}
type any = Any : _ -> any
let escape (Any x) = <<x>>
\end{camlinput}
\begin{camlerror}
Error: This expression has type $\textdollar$Any but an expression was expected of type
         'a
       The type constructor $\textdollar$Any would escape its scope
\end{camlerror}
\end{caml}
\end{camlexample}
\item "$'a" if the existential variable was unified with the type %$
variable "'a" during typing:
\begin{camlexample}{verbatim}
\begin{caml}
\begin{camlinput}
type ('arg,'result,'aux) fn =
  | Fun: ('a ->'b) -> ('a,'b,unit) fn
  | Mem1: ('a ->'b) * 'a * 'b -> ('a, 'b, 'a * 'b) fn
 let apply: ('arg,'result, _ ) fn -> 'arg -> 'result = fun f x ->
  match f with
  | Fun f -> f x
  | <<Mem1 (f,y,fy)>> -> if x = y then fy else f x
\end{camlinput}
\begin{camlerror}
Error: This pattern matches values of type
         ($\textdollar$'arg, 'result, $\textdollar$'arg * 'result) fn
       but a pattern was expected which matches values of type
         ($\textdollar$'arg, 'result, unit) fn
       The type constructor $\textdollar$'arg would escape its scope
\end{camlerror}
\end{caml}
\end{camlexample}
\item "$n" (n a number) is an internally generated existential %$
which could not be named using one of the previous schemes.
\end{itemize}

As shown by the last item, the current behavior is imperfect
and may be improved in future versions.

\lparagraph{p:explicit-existential-name}{Explicit naming of
  existentials} (Introduced in OCaml 4.13.0)

As explained above, pattern-matching on a GADT constructor may
introduce existential types. The following syntax allows to name them
explicitly.

\begin{syntax}
pattern:
     ...
   | constr '(' "type" {{typeconstr-name}} ')' '(' pattern ')'
;
\end{syntax}

For instance, the following code names the type of the argument of f
and uses this name.

\begin{caml_example*}{verbatim}
type _ closure = Closure : ('a -> 'b) * 'a -> 'b closure
let eval = fun (Closure (type a) (f, x : (a -> _) * _)) -> f (x : a)
\end{caml_example*}
All existential type variables of the constructor must by introduced by
the ("type" ...) construct and bound by a type annotation on the
outside of the constructor argument.

\lparagraph{p:gadt-equation-nonlocal-abstract}{Equations on non-local abstract types} (Introduced in OCaml
4.04)

GADT pattern-matching may also add type equations to non-local
abstract types. The behaviour is the same as with local abstract
types. Reusing the above "eq" type, one can write:
\begin{camlexample}{verbatim}
\begin{caml}
\begin{camlinput}
module M : sig type t val x : t val e : (t,int) eq end = struct
  type t = int
  let x = 33
  let e = Eq
end

let x : int = let Eq = M.e in M.x
\end{camlinput}
\end{caml}
\end{camlexample}

Of course, not all abstract types can be refined, as this would
contradict the exhaustiveness check. Namely, builtin types (those
defined by the compiler itself, such as "int" or "array"), and
abstract types defined by the local module, are non-instantiable, and
as such cause a type error rather than introduce an equation.

\section{s:bigarray-access}{Syntax for Bigarray access}
%HEVEA\cutname{bigarray.html}

(Introduced in Objective Caml 3.00)

\begin{syntax}
expr:
          ...
        | expr '.{' expr { ',' expr } '}'
        | expr '.{' expr { ',' expr } '}' '<-' expr
\end{syntax}

This extension provides syntactic sugar for getting and setting
elements in the arrays provided by the \stdmoduleref{Bigarray} module.

The short expressions are translated into calls to functions of the
"Bigarray" module as described in the following table.

\begin{tableau}{|l|l|}{expression}{translation}
\entree{@expr_0'.{'expr_1'}'@}
       {"Bigarray.Array1.get "@expr_0 expr_1@}
\entree{@expr_0'.{'expr_1'}' '<-'expr@}
       {"Bigarray.Array1.set "@expr_0 expr_1 expr@}
\entree{@expr_0'.{'expr_1',' expr_2'}'@}
       {"Bigarray.Array2.get "@expr_0 expr_1 expr_2@}
\entree{@expr_0'.{'expr_1',' expr_2'}' '<-'expr@}
       {"Bigarray.Array2.set "@expr_0 expr_1 expr_2 expr@}
\entree{@expr_0'.{'expr_1',' expr_2',' expr_3'}'@}
       {"Bigarray.Array3.get "@expr_0 expr_1 expr_2 expr_3@}
\entree{@expr_0'.{'expr_1',' expr_2',' expr_3'}' '<-'expr@}
       {"Bigarray.Array3.set "@expr_0 expr_1 expr_2 expr_3 expr@}
\entree{@expr_0'.{'expr_1',' \ldots',' expr_n'}'@}
       {"Bigarray.Genarray.get "@ expr_0 '[|' expr_1',' \ldots ','
        expr_n '|]'@}
\entree{@expr_0'.{'expr_1',' \ldots',' expr_n'}' '<-'expr@}
       {"Bigarray.Genarray.set "@ expr_0 '[|' expr_1',' \ldots ','
        expr_n '|]' expr@}
\end{tableau}

The last two entries are valid for any $n > 3$.

\section{s:attributes}{Attributes}
%HEVEA\cutname{attributes.html}

\ikwd{when\@\texttt{when}}

(Introduced in OCaml 4.02,
infix notations for constructs other than expressions added in 4.03)

Attributes are ``decorations'' of the syntax tree which are mostly
ignored by the type-checker but can be used by external tools.  An
attribute is made of an identifier and a payload, which can be a
structure, a type expression (prefixed with ":"), a signature
(prefixed with ":") or a pattern (prefixed with "?") optionally
followed by a "when" clause:


\begin{syntax}
attr-id:
    lowercase-ident
 |  capitalized-ident
 |  attr-id '.' attr-id
;
attr-payload:
    [ module-items ]
 |  ':' typexpr
 |  ':' [ specification ]
 |  '?' pattern ['when' expr]
;
\end{syntax}

The first form of attributes is attached with a postfix notation on
``algebraic'' categories:

\begin{syntax}
attribute:
    '[@' attr-id attr-payload ']'
;
expr: ...
     | expr attribute
;
typexpr: ...
     | typexpr attribute
;
pattern: ...
     | pattern attribute
;
module-expr: ...
     | module-expr attribute
;
module-type: ...
     | module-type attribute
;
class-expr: ...
     | class-expr attribute
;
class-type: ...
     | class-type attribute
;
\end{syntax}

This form of attributes can also be inserted after the @'`'tag-name@
in polymorphic variant type expressions (@tag-spec-first@, @tag-spec@,
@tag-spec-full@) or after the @method-name@ in @method-type@.

The same syntactic form is also used to attach attributes to labels and
constructors in type declarations:

\begin{syntax}
field-decl:
          ['mutable'] field-name ':' poly-typexpr {attribute}
;
constr-decl:
          (constr-name || '()') [ 'of' constr-args ] {attribute}
;
\end{syntax}

Note: when a label declaration is followed by a semi-colon, attributes
can also be put after the semi-colon (in which case they are merged to
those specified before).


The second form of attributes are attached to ``blocks'' such as type
declarations, class fields, etc:

\begin{syntax}
item-attribute:
    '[@@' attr-id attr-payload ']'
;
typedef: ...
   | typedef item-attribute
;
exception-definition:
        'exception' constr-decl
      | 'exception' constr-name '=' constr
;
module-items:
        [';;'] ( definition || expr { item-attribute } ) { [';;'] definition || ';;' expr { item-attribute } } [';;']
;
class-binding: ...
   | class-binding item-attribute
;
class-spec: ...
   | class-spec item-attribute
;
classtype-def: ...
   | classtype-def item-attribute
;
definition:
          'let' ['rec'] let-binding { 'and' let-binding }
        | 'external' value-name ':' typexpr '=' external-declaration { item-attribute }
        | type-definition
        | exception-definition { item-attribute }
        | class-definition
        | classtype-definition
        | 'module' module-name { '(' module-name ':' module-type ')' }
                   [ ':' module-type ] \\ '=' module-expr { item-attribute }
        | 'module' 'type' modtype-name '=' module-type { item-attribute }
        | 'open' module-path { item-attribute }
        | 'include' module-expr { item-attribute }
        | 'module' 'rec' module-name ':' module-type '=' \\
          module-expr { item-attribute } \\
          { 'and' module-name ':' module-type '=' module-expr \\
          { item-attribute } }
;
specification:
          'val' value-name ':' typexpr { item-attribute }
        | 'external' value-name ':' typexpr '=' external-declaration { item-attribute }
        | type-definition
        | 'exception' constr-decl { item-attribute }
        | class-specification
        | classtype-definition
        | 'module' module-name ':' module-type { item-attribute }
        | 'module' module-name { '(' module-name ':' module-type ')' }
          ':' module-type { item-attribute }
        | 'module' 'type' modtype-name { item-attribute }
        | 'module' 'type' modtype-name '=' module-type { item-attribute }
        | 'open' module-path { item-attribute }
        | 'include' module-type { item-attribute }
;
class-field-spec: ...
        | class-field-spec item-attribute
;
class-field: ...
        | class-field item-attribute
;
\end{syntax}

A third form of attributes appears as stand-alone structure or
signature items in the module or class sub-languages.  They are not
attached to any specific node in the syntax tree:

\begin{syntax}
floating-attribute:
    '[@@@' attr-id attr-payload ']'
;
definition: ...
   | floating-attribute
;
specification: ...
   | floating-attribute
;
class-field-spec: ...
   | floating-attribute
;
class-field: ...
   | floating-attribute
;
\end{syntax}

(Note: contrary to what the grammar above describes, @item-attributes@
cannot be attached to these floating attributes in @class-field-spec@
and @class-field@.)


It is also possible to specify attributes using an infix syntax. For instance:

\begin{verbatim}
let[@foo] x = 2 in x + 1          === (let x = 2 [@@foo] in x + 1)
begin[@foo][@bar x] ... end       === (begin ... end)[@foo][@bar x]
module[@foo] M = ...              === module M = ... [@@foo]
type[@foo] t = T                  === type t = T [@@foo]
method[@foo] m = ...              === method m = ... [@@foo]
\end{verbatim}

For "let", the attributes are applied to each bindings:

\begin{verbatim}
let[@foo] x = 2 and y = 3 in x + y === (let x = 2 [@@foo] and y = 3 in x + y)
let[@foo] x = 2
and[@bar] y = 3 in x + y           === (let x = 2 [@@foo] and y = 3 [@@bar] in x + y)
\end{verbatim}


\subsection{ss:builtin-attributes}{Built-in attributes}

Some attributes are understood by the type-checker:
\begin{itemize}
\item
 ``ocaml.warning'' or ``warning'', with a string literal payload.
 This can be used as floating attributes in a
 signature/structure/object/object type.  The string is parsed and has
 the same effect as the "-w" command-line option, in the scope between
 the attribute and the end of the current
 signature/structure/object/object type.  The attribute can also be
 attached to any kind of syntactic item which support attributes
 (such as an expression, or a type expression)
 in which case its scope is limited to that item.
 Note that it is not well-defined which scope is used for a specific
 warning.  This is implementation dependent and can change between versions.
 Some warnings are even completely outside the control of ``ocaml.warning''
 (for instance, warnings 1, 2, 14, 29 and 50).

\item
 ``ocaml.warnerror'' or ``warnerror'', with a string literal payload.
 Same as ``ocaml.warning'', for the "-warn-error" command-line option.

\item
 ``ocaml.alert'' or ``alert'': see section~\ref{s:alerts}.

\item
  ``ocaml.deprecated'' or ``deprecated'': alias for the
  ``deprecated'' alert, see section~\ref{s:alerts}.
\item
  ``ocaml.deprecated_mutable'' or ``deprecated_mutable''.
  Can be applied to a mutable record label.  If the label is later
  used to modify the field (with ``expr.l <- expr''), the ``deprecated'' alert
  will be triggered.  If the payload of the attribute is a string literal,
  the alert message includes this text.
\item
  ``ocaml.ppwarning'' or ``ppwarning'', in any context, with
  a string literal payload.  The text is reported as warning (22)
  by the compiler (currently, the warning location is the location
  of the string payload).  This is mostly useful for preprocessors which
  need to communicate warnings to the user.  This could also be used
  to mark explicitly some code location for further inspection.
\item
  ``ocaml.warn_on_literal_pattern'' or ``warn_on_literal_pattern'' annotate
  constructors in type definition. A warning (52) is then emitted when this
  constructor is pattern matched with a constant literal as argument. This
  attribute denotes constructors whose argument is purely informative and
  may change in the future. Therefore, pattern matching on this argument
  with a constant literal is unreliable. For instance, all built-in exception
  constructors are marked as ``warn_on_literal_pattern''.
  Note that, due to an implementation limitation, this warning (52) is only
  triggered for single argument constructor.
\item
  ``ocaml.tailcall'' or ``tailcall'' can be applied to function
  application in order to check that the call is tailcall optimized.
  If it it not the case, a warning (51) is emitted.
\item
  ``ocaml.inline'' or ``inline'' take either ``never'', ``always''
  or nothing as payload on a function or functor definition. If no payload
  is provided, the default value is ``always''. This payload controls when
  applications of the annotated functions should be inlined.
\item
  ``ocaml.inlined'' or ``inlined'' can be applied to any function or functor
  application to check that the call is inlined by the compiler. If the call
  is not inlined, a warning (55) is emitted.
\item
  ``ocaml.noalloc'', ``ocaml.unboxed''and ``ocaml.untagged'' or
  ``noalloc'', ``unboxed'' and ``untagged'' can be used on external
  definitions to obtain finer control over the C-to-OCaml interface. See
  \ref{s:C-cheaper-call} for more details.
\item
  ``ocaml.immediate'' or ``immediate'' applied on an abstract type mark the type as
  having a non-pointer implementation (e.g. ``int'', ``bool'', ``char'' or
  enumerated types). Mutation of these immediate types does not activate the
  garbage collector's write barrier, which can significantly boost performance in
  programs relying heavily on mutable state.
\item
  ``ocaml.immediate64'' or ``immediate64'' applied on an abstract type mark the
  type as having a non-pointer implementation on 64 bit platforms. No assumption
  is made on other platforms. In order to produce a type with the
  ``immediate64`` attribute, one must use ``Sys.Immediate64.Make`` functor.
\item
  "ocaml.unboxed" or "unboxed" can be used on a type definition if the
  type is a single-field record or a concrete type with a single
  constructor that has a single argument. It tells the compiler to
  optimize the representation of the type by removing the block that
  represents the record or the constructor (i.e. a value of this type
  is physically equal to its argument). In the case of GADTs, an
  additional restriction applies: the argument must not be an
  existential variable, represented by an existential type variable,
  or an abstract type constructor applied to an existential type
  variable.
\item
   "ocaml.boxed" or "boxed" can be used on type definitions to mean
   the opposite of "ocaml.unboxed": keep the unoptimized
   representation of the type. When there is no annotation, the
   default is currently "boxed" but it may change in the future.
 \item
   "ocaml.local" or "local" take either "never", "always", "maybe" or
   nothing as payload on a function definition.  If no payload is
   provided, the default is "always".  The attribute controls an
   optimization which consists in compiling a function into a static
   continuation.  Contrary to inlining, this optimization does not
   duplicate the function's body.  This is possible when all
   references to the function are full applications, all sharing the
   same continuation (for instance, the returned value of several
   branches of a pattern matching). "never" disables the optimization,
   "always" asserts that the optimization applies (otherwise a warning
   55 is emitted) and "maybe" lets the optimization apply when
   possible (this is the default behavior when the attribute is not
   specified).  The optimization is implicitly disabled when using the
   bytecode compiler in debug mode (-g), and for functions marked with
   an "ocaml.inline always" or "ocaml.unrolled" attribute which
   supersede "ocaml.local".
\end{itemize}

\begin{camlexample}{verbatim}
\begin{caml}
\begin{camlinput}
module X = struct
  [@@@warning "+9"]  (* locally enable warning 9 in this structure *)
  $\ldots$
end
[@@deprecated "Please use module 'Y' instead."]

let x = begin[@warning "+9"] [$\ldots$] end

type t = A | B
  [@@deprecated "Please use type 's' instead."]
\end{camlinput}
\end{caml}
\end{camlexample}

\begin{camlexample}{verbatim}
\begin{caml}
\begin{camlinput}
let fires_warning_22 x =
  assert (x >= 0) [@ppwarning <<"TODO: remove this later">>]
\end{camlinput}
\begin{camlwarn}
Warning 22 [preprocessor]: TODO: remove this later
\end{camlwarn}
\end{caml}
\end{camlexample}

\begin{camlexample}{verbatim}
\begin{caml}
\begin{camlinput}
let rec is_a_tail_call = function
  | [] -> ()
  | _ :: q -> (is_a_tail_call[@tailcall]) q

let rec not_a_tail_call = function
  | [] -> []
  | x :: q -> x :: <<(not_a_tail_call[@tailcall]) q>>
\end{camlinput}
\begin{camlwarn}
Warning 51 [wrong-tailcall-expectation]: expected tailcall
\end{camlwarn}
\end{caml}
\end{camlexample}

\begin{camlexample}{verbatim}
\begin{caml}
\begin{camlinput}
let f x = x [@@inline]

let () = (f[@inlined]) ()
\end{camlinput}
\end{caml}
\end{camlexample}

\begin{camlexample}{verbatim}
\begin{caml}
\begin{camlinput}
type fragile =
  | Int of int [@warn_on_literal_pattern]
  | String of string [@warn_on_literal_pattern]
\end{camlinput}
\end{caml}
\end{camlexample}

\begin{camlexample}{verbatim}
\begin{caml}
\begin{camlinput}
let fragile_match_1 = function
| Int <<0>> -> ()
| _ -> ()
\end{camlinput}
\begin{camlwarn}
Warning 52 [fragile-literal-pattern]: Code should not depend on the actual values of
this constructor's arguments. They are only for information
and may change in future versions. (See manual section 9.5)
val fragile_match_1 : fragile -> unit = <fun>
\end{camlwarn}
\end{caml}
\end{camlexample}

\begin{camlexample}{verbatim}
\begin{caml}
\begin{camlinput}
let fragile_match_2 = function
| String <<"constant">> -> ()
| _ -> ()
\end{camlinput}
\begin{camlwarn}
Warning 52 [fragile-literal-pattern]: Code should not depend on the actual values of
this constructor's arguments. They are only for information
and may change in future versions. (See manual section 9.5)
val fragile_match_2 : fragile -> unit = <fun>
\end{camlwarn}
\end{caml}
\end{camlexample}

\begin{camlexample}{verbatim}
\begin{caml}
\begin{camlinput}
module Immediate: sig
  type t [@@immediate]
  val x: t ref
end = struct
  type t = A | B
  let x = ref A
end
\end{camlinput}
\end{caml}
\end{camlexample}

\begin{camlexample}{verbatim}
\begin{caml}
\begin{camlinput}
module Int_or_int64 : sig
  type t [@@immediate64]
  val zero : t
  val one : t
  val add : t -> t -> t
end = struct

  include Sys.Immediate64.Make(Int)(Int64)

  module type S = sig
    val zero : t
    val one : t
    val add : t -> t -> t
  end

  let impl : (module S) =
    match repr with
    | Immediate ->
        (module Int : S)
    | Non_immediate ->
        (module Int64 : S)

  include (val impl : S)
end
\end{camlinput}
\end{caml}
\end{camlexample}

\section{s:extension-nodes}{Extension nodes}
%HEVEA\cutname{extensionnodes.html}

(Introduced in OCaml 4.02,
infix notations for constructs other than expressions added in 4.03,
infix notation (e1 ;\%ext e2) added in 4.04.
)

Extension nodes are generic placeholders in the syntax tree. They are
rejected by the type-checker and are intended to be ``expanded'' by external
tools such as "-ppx" rewriters.

Extension nodes share the same notion of identifier and payload as
attributes~\ref{s:attributes}.

The first form of extension node is used for ``algebraic'' categories:

\begin{syntax}
extension:
    '[%' attr-id attr-payload ']'
;
expr: ...
     | extension
;
typexpr: ...
     | extension
;
pattern: ...
     | extension
;
module-expr: ...
     | extension
;
module-type: ...
     | extension
;
class-expr: ...
     | extension
;
class-type: ...
     | extension
;
\end{syntax}

A second form of extension node can be used in structures and
signatures, both in the module and object languages:

\begin{syntax}
item-extension:
    '[%%' attr-id attr-payload ']'
;
definition: ...
   | item-extension
;
specification: ...
   | item-extension
;
class-field-spec: ...
   | item-extension
;
class-field: ...
   | item-extension
;
\end{syntax}

An infix form is available for extension nodes when
the payload is of the same kind
(expression with expression, pattern with pattern ...).

Examples:

\begin{verbatim}
let%foo x = 2 in x + 1     === [%foo let x = 2 in x + 1]
begin%foo ... end          === [%foo begin ... end]
x ;%foo 2                  === [%foo x; 2]
module%foo M = ..          === [%%foo module M = ... ]
val%foo x : t              === [%%foo: val x : t]
\end{verbatim}

When this form is used together with the infix syntax for attributes,
the attributes are considered to apply to the payload:

\begin{verbatim}
fun%foo[@bar] x -> x + 1 === [%foo (fun x -> x + 1)[@bar ] ];
\end{verbatim}

Furthermore, quoted strings "{|...|}" can be combined with extension nodes
to embed foreign syntax fragments. Those fragments can be interpreted
by a preprocessor and turned into OCaml code without requiring escaping
quotes. A syntax shortcut is available for them:

\begin{verbatim}
{%%foo|...|}               === [%%foo{|...|}]
let x = {%foo|...|}        === let x = [%foo{|...|}]
let y = {%foo bar|...|bar} === let y = [%foo{bar|...|bar}]
\end{verbatim}

For instance, you can use "{%sql|...|}" to
represent arbitrary SQL statements -- assuming you have a ppx-rewriter
that recognizes the "%sql" extension.

Note that the word-delimited form, for example "{sql|...|sql}", should
not be used for signaling that an extension is in use.
Indeed, the user cannot see from the code whether this string literal has
different semantics than they expect. Moreover, giving semantics to a
specific delimiter limits the freedom to change the delimiter to avoid
escaping issues.

\subsection{ss:builtin-extension-nodes}{Built-in extension nodes}

(Introduced in OCaml 4.03)

Some extension nodes are understood by the compiler itself:
\begin{itemize}
  \item
    ``ocaml.extension_constructor'' or ``extension_constructor''
    take as payload a constructor from an extensible variant type
    (see \ref{s:extensible-variants}) and return its extension
    constructor slot.
\end{itemize}

\begin{camlexample}{verbatim}
\begin{caml}
\begin{camlinput}
type t = ..
type t += X of int | Y of string
let x = [%extension_constructor X]
let y = [%extension_constructor Y]
\end{camlinput}
\end{caml}
\end{camlexample}
\begin{camlexample}{toplevel}
\begin{caml}
\begin{camlinput}
$\?$  x <> y;;
\end{camlinput}
\begin{camloutput}
- : bool = true
\end{camloutput}
\end{caml}
\end{camlexample}

\section{s:extensible-variants}{Extensible variant types}
%HEVEA\cutname{extensiblevariants.html}

(Introduced in OCaml 4.02)

\begin{syntax}
type-representation:
          ...
        | '=' '..'
;
specification:
        ...
      | 'type' [type-params] typeconstr type-extension-spec
;
definition:
        ...
      | 'type' [type-params] typeconstr type-extension-def
;
type-extension-spec: '+=' ['private'] ['|'] constr-decl { '|' constr-decl }
;
type-extension-def: '+=' ['private'] ['|'] constr-def { '|' constr-def }
;
constr-def:
          constr-decl
        | constr-name '=' constr
;
\end{syntax}

Extensible variant types are variant types which can be extended with
new variant constructors. Extensible variant types are defined using
"..". New variant constructors are added using "+=".
\begin{camlexample}{verbatim}
\begin{caml}
\begin{camlinput}
module Expr = struct
  type attr = ..

  type attr += Str of string

  type attr +=
    | Int of int
    | Float of float
end
\end{camlinput}
\end{caml}
\end{camlexample}

Pattern matching on an extensible variant type requires a default case
to handle unknown variant constructors:
\begin{camlexample}{verbatim}
\begin{caml}
\begin{camlinput}
let to_string = function
  | Expr.Str s -> s
  | Expr.Int i -> Int.to_string i
  | Expr.Float f -> string_of_float f
  | _ -> "?"
\end{camlinput}
\end{caml}
\end{camlexample}

A preexisting example of an extensible variant type is the built-in
"exn" type used for exceptions. Indeed, exception constructors can be
declared using the type extension syntax:
\begin{camlexample}{verbatim}
\begin{caml}
\begin{camlinput}
type exn += Exc of int
\end{camlinput}
\end{caml}
\end{camlexample}

Extensible variant constructors can be rebound to a different name. This
allows exporting variants from another module.
\begin{camlexample}{toplevel}
\begin{caml}
\begin{camlinput}
$\?$ let not_in_scope = <<Str>> "Foo";;
\end{camlinput}
\begin{camlerror}
Error: Unbound constructor Str
\end{camlerror}
\end{caml}
\end{camlexample}
\begin{camlexample}{verbatim}
\begin{caml}
\begin{camlinput}
type Expr.attr += Str = Expr.Str
\end{camlinput}
\end{caml}
\end{camlexample}
\begin{camlexample}{toplevel}
\begin{caml}
\begin{camlinput}
$\?$ let now_works = Str "foo";;
\end{camlinput}
\begin{camloutput}
val now_works : Expr.attr = Expr.Str "foo"
\end{camloutput}
\end{caml}
\end{camlexample}

Extensible variant constructors can be declared "private". As with
regular variants, this prevents them from being constructed directly by
constructor application while still allowing them to be de-structured in
pattern-matching.
\begin{camlexample}{verbatim}
\begin{caml}
\begin{camlinput}
module B : sig
  type Expr.attr += private Bool of int
  val bool : bool -> Expr.attr
end = struct
  type Expr.attr += Bool of int
  let bool p = if p then Bool 1 else Bool 0
end
\end{camlinput}
\end{caml}
\end{camlexample}

\begin{camlexample}{toplevel}
\begin{caml}
\begin{camlinput}
$\?$ let inspection_works = function
    | B.Bool p -> (p = 1)
    | _ -> true;;
\end{camlinput}
\begin{camloutput}
val inspection_works : Expr.attr -> bool = <fun>
\end{camloutput}
\end{caml}
\end{camlexample}
\begin{camlexample}{toplevel}
\begin{caml}
\begin{camlinput}
$\?$ let construction_is_forbidden = <<B.Bool 1>>;;
\end{camlinput}
\begin{camlerror}
Error: Cannot use private constructor Bool to create values of type Expr.attr
\end{camlerror}
\end{caml}
\end{camlexample}

\subsection{ss:private-extensible}{Private extensible variant types}

(Introduced in OCaml 4.06)

\begin{syntax}
type-representation:
          ...
        | '=' 'private' '..'
;
\end{syntax}

Extensible variant types can be declared "private". This prevents new
constructors from being declared directly, but allows extension
constructors to be referred to in interfaces.
\begin{camlexample}{verbatim}
\begin{caml}
\begin{camlinput}
module Msg : sig
  type t = private ..
  module MkConstr (X : sig type t end) : sig
    type t += C of X.t
  end
end = struct
  type t = ..
  module MkConstr (X : sig type t end) = struct
    type t += C of X.t
  end
end
\end{camlinput}
\end{caml}
\end{camlexample}

\section{s:generative-functors}{Generative functors}
%HEVEA\cutname{generativefunctors.html}

(Introduced in OCaml 4.02)

\begin{syntax}
module-expr:
          ...
        | 'functor' '()' '->' module-expr
        | module-expr '()'
;
definition:
          ...
        | 'module' module-name { '(' module-name ':' module-type ')' || '()' }
                   [ ':' module-type ] \\ '=' module-expr
;
module-type:
          ...
        | 'functor' '()' '->' module-type
;
specification:
          ...
        | 'module' module-name { '(' module-name ':' module-type ')' || '()' }
          ':' module-type
;
\end{syntax}

A generative functor takes a unit "()" argument.
In order to use it, one must necessarily apply it to this unit argument,
ensuring that all type components in the result of the functor behave
in a generative way, {\em i.e.} they are different from types obtained
by other applications of the same functor.
This is equivalent to taking an argument of signature "sig end", and always
applying to "struct end", but not to some defined module (in the
latter case, applying twice to the same module would return identical
types).

As a side-effect of this generativity, one is allowed to unpack
first-class modules in the body of generative functors.

\section{s:extension-syntax}{Extension-only syntax}
%HEVEA\cutname{extensionsyntax.html}
(Introduced in OCaml 4.02.2, extended in 4.03)

Some syntactic constructions are accepted during parsing and rejected
during type checking. These syntactic constructions can therefore not
be used directly in vanilla OCaml. However, "-ppx" rewriters and other
external tools can exploit this parser leniency to extend the language
with these new syntactic constructions by rewriting them to
vanilla constructions.
\subsection{ss:extension-operators}{Extension operators} \label{s:ext-ops}
(Introduced in OCaml 4.02.2)
\begin{syntax}
infix-symbol:
          ...
        | "#" {operator-chars} "#"  {operator-char '|' "#"}
;
\end{syntax}

Operator names starting with a "#" character and containing more than
one "#" character are reserved for extensions.

\subsection{ss:extension-literals}{Extension literals}
(Introduced in OCaml 4.03)
\begin{syntax}
float-literal:
       ...
     | ["-"] ("0"\ldots"9") { "0"\ldots"9"||"_" } ["." { "0"\ldots"9"||"_" }]
       [("e"||"E") ["+"||"-"] ("0"\ldots"9") { "0"\ldots"9"||"_" }]
       ["g"\ldots"z"||"G"\ldots"Z"]
     | ["-"] ("0x"||"0X")
       ("0"\ldots"9"||"A"\ldots"F"||"a"\ldots"f")
       { "0"\ldots"9"||"A"\ldots"F"||"a"\ldots"f"||"_" }\\
       ["." { "0"\ldots"9"||"A"\ldots"F"||"a"\ldots"f"||"_" }]
       [("p"||"P") ["+"||"-"] ("0"\ldots"9") { "0"\ldots"9"||"_" }]
       ["g"\ldots"z"||"G"\ldots"Z"]
;
int-literal:
           ...
        | ["-"] ("0"\ldots"9") { "0"\ldots"9" || "_" }["g"\ldots"z"||"G"\ldots"Z"]
        | ["-"] ("0x"||"0X") ("0"\ldots"9"||"A"\ldots"F"||"a"\ldots"f")
          { "0"\ldots"9"||"A"\ldots"F"||"a"\ldots"f"||"_" }
          ["g"\ldots"z"||"G"\ldots"Z"]
        | ["-"] ("0o"||"0O") ("0"\ldots"7") { "0"\ldots"7"||"_" }
          ["g"\ldots"z"||"G"\ldots"Z"]
        | ["-"] ("0b"||"0B") ("0"\ldots"1") { "0"\ldots"1"||"_" }
          ["g"\ldots"z"||"G"\ldots"Z"]
;
\end{syntax}
Int and float literals followed by an one-letter identifier in the
range @["g".."z"||"G".."Z"]@ are extension-only literals.

\section{s:inline-records}{Inline records}
%HEVEA\cutname{inlinerecords.html}
(Introduced in OCaml 4.03)
\begin{syntax}
  constr-args:
          ...
          | record-decl
;
\end{syntax}

The arguments of sum-type constructors can now be defined using the
same syntax as records.  Mutable and polymorphic fields are allowed.
GADT syntax is supported.  Attributes can be specified on individual
fields.

Syntactically, building or matching constructors with such an inline
record argument is similar to working with a unary constructor whose
unique argument is a declared record type.  A pattern can bind
the inline record as a pseudo-value, but the record cannot escape the
scope of the binding and can only be used with the dot-notation to
extract or modify fields or to build new constructor values.

\begin{camlexample}{verbatim}
\begin{caml}
\begin{camlinput}
type t =
  | Point of {width: int; mutable x: float; mutable y: float}
  | Other

let v = Point {width = 10; x = 0.; y = 0.}

let scale l = function
  | Point p -> Point {p with x = l *. p.x; y = l *. p.y}
  | Other -> Other

let print = function
  | Point {x; y; _} -> Printf.printf "%f/%f" x y
  | Other -> ()

let reset = function
  | Point p -> p.x <- 0.; p.y <- 0.
  | Other -> ()
\end{camlinput}
\end{caml}
\end{camlexample}

\begin{camlexample}{verbatim}
\begin{caml}
\begin{camlinput}
let invalid = function
  | Point p -> <<p>>
\end{camlinput}
\begin{camlerror}
Error: This form is not allowed as the type of the inlined record could escape.
\end{camlerror}
\end{caml}
\end{camlexample}

\section{s:doc-comments}{Documentation comments}
%HEVEA\cutname{doccomments.html}
(Introduced in OCaml 4.03)

Comments which start with "**" are treated specially by the
compiler. They are automatically converted during parsing into
attributes (see \ref{s:attributes}) to allow tools to process them as
documentation.

Such comments can take three forms: {\em floating comments}, {\em item
comments} and {\em label comments}. Any comment starting with "**" which
does not match one of these forms will cause the compiler to emit
warning 50.

Comments which start with "**" are also used by the ocamldoc
documentation generator (see \ref{c:ocamldoc}). The three comment forms
recognised by the compiler are a subset of the forms accepted by
ocamldoc (see \ref{s:ocamldoc-comments}).

\subsection{ss:floating-comments}{Floating comments}

Comments surrounded by blank lines that appear within structures,
signatures, classes or class types are converted into
@floating-attribute@s. For example:

\begin{camlexample}{verbatim}
\begin{caml}
\begin{camlinput}
type t = T

(** Now some definitions for [t] *)

let mkT = T
\end{camlinput}
\end{caml}
\end{camlexample}

will be converted to:

\begin{camlexample}{verbatim}
\begin{caml}
\begin{camlinput}
type t = T

[@@@ocaml.text " Now some definitions for [t] "]

let mkT = T
\end{camlinput}
\end{caml}
\end{camlexample}

\subsection{ss:item-comments}{Item comments}

Comments which appear {\em immediately before} or {\em immediately
after} a structure item, signature item, class item or class type item
are converted into @item-attribute@s. Immediately before or immediately
after means that there must be no blank lines, ";;", or other
documentation comments between them. For example:

\begin{camlexample}{verbatim}
\begin{caml}
\begin{camlinput}
type t = T
(** A description of [t] *)

\end{camlinput}
\end{caml}
\end{camlexample}

or

\begin{camlexample}{verbatim}
\begin{caml}
\begin{camlinput}
(** A description of [t] *)
type t = T
\end{camlinput}
\end{caml}
\end{camlexample}

will be converted to:

\begin{camlexample}{verbatim}
\begin{caml}
\begin{camlinput}
type t = T
[@@ocaml.doc " A description of [t] "]
\end{camlinput}
\end{caml}
\end{camlexample}

Note that, if a comment appears immediately next to multiple items,
as in:

\begin{camlexample}{verbatim}
\begin{caml}
\begin{camlinput}
type t = T
(** An ambiguous comment *)
type s = S
\end{camlinput}
\end{caml}
\end{camlexample}

then it will be attached to both items:

\begin{camlexample}{verbatim}
\begin{caml}
\begin{camlinput}
type t = T
[@@ocaml.doc " An ambiguous comment "]
type s = S
[@@ocaml.doc " An ambiguous comment "]
\end{camlinput}
\end{caml}
\end{camlexample}

and the compiler will emit warning 50.

\subsection{ss:label-comments}{Label comments}

Comments which appear {\em immediately after} a labelled argument,
record field, variant constructor, object method or polymorphic variant
constructor are are converted into @attribute@s. Immediately
after means that there must be no blank lines or other documentation
comments between them. For example:

\begin{camlexample}{verbatim}
\begin{caml}
\begin{camlinput}
type t1 = lbl:int (** Labelled argument *) -> unit

type t2 = {
  fld: int; (** Record field *)
  fld2: float;
}

type t3 =
  | Cstr of string (** Variant constructor *)
  | Cstr2 of string

type t4 = < meth: int * int; (** Object method *) >

type t5 = [
  `PCstr (** Polymorphic variant constructor *)
]
\end{camlinput}
\end{caml}
\end{camlexample}

will be converted to:

\begin{camlexample}{verbatim}
\begin{caml}
\begin{camlinput}
type t1 = lbl:(int [@ocaml.doc " Labelled argument "]) -> unit

type t2 = {
  fld: int [@ocaml.doc " Record field "];
  fld2: float;
}

type t3 =
  | Cstr of string [@ocaml.doc " Variant constructor "]
  | Cstr2 of string

type t4 = < meth : int * int [@ocaml.doc " Object method "] >

type t5 = [
  `PCstr [@ocaml.doc " Polymorphic variant constructor "]
]
\end{camlinput}
\end{caml}
\end{camlexample}

Note that label comments take precedence over item comments, so:

\begin{camlexample}{verbatim}
\begin{caml}
\begin{camlinput}
type t = T of string
(** Attaches to T not t *)
\end{camlinput}
\end{caml}
\end{camlexample}

will be converted to:

\begin{camlexample}{verbatim}
\begin{caml}
\begin{camlinput}
type t =  T of string [@ocaml.doc " Attaches to T not t "]
\end{camlinput}
\end{caml}
\end{camlexample}

whilst:

\begin{camlexample}{verbatim}
\begin{caml}
\begin{camlinput}
type t = T of string
(** Attaches to T not t *)
(** Attaches to t *)
\end{camlinput}
\end{caml}
\end{camlexample}

will be converted to:

\begin{camlexample}{verbatim}
\begin{caml}
\begin{camlinput}
type t =  T of string [@ocaml.doc " Attaches to T not t "]
[@@ocaml.doc " Attaches to t "]
\end{camlinput}
\end{caml}
\end{camlexample}

In the absence of meaningful comment on the last constructor of
a type, an empty comment~"(**)" can be used instead:

\begin{camlexample}{verbatim}
\begin{caml}
\begin{camlinput}
type t = T of string
(**)
(** Attaches to t *)
\end{camlinput}
\end{caml}
\end{camlexample}

will be converted directly to

\begin{camlexample}{verbatim}
\begin{caml}
\begin{camlinput}
type t =  T of string
[@@ocaml.doc " Attaches to t "]
\end{camlinput}
\end{caml}
\end{camlexample}

\section{s:index-operators}{Extended indexing operators }
%HEVEA\cutname{indexops.html}
(Introduced in 4.06)

\begin{syntax}

dot-ext:
   | dot-operator-char { operator-char }
;
dot-operator-char:
  '!' ||  '?' || core-operator-char || '%' || ':'
;
expr:
          ...
        | expr '.' [module-path '.'] dot-ext ( '(' expr ')' || '[' expr ']' || '{' expr '}' ) [ '<-' expr ]
;
operator-name:
          ...
        | '.' dot-ext ('()' || '[]' || '{}') ['<-']
;
\end{syntax}


This extension provides syntactic sugar for getting and setting elements
for user-defined indexed types. For instance, we can define python-like
dictionaries with
\begin{camlexample}{verbatim}
\begin{caml}
\begin{camlinput}
module Dict = struct
include Hashtbl
let ( .%{} ) tabl index = find tabl index
let ( .%{}<- ) tabl index value = add tabl index value
end
let dict =
  let dict = Dict.create 10 in
  let () =
    dict.Dict.%{"one"} <- 1;
    let open Dict in
    dict.%{"two"} <- 2 in
  dict
\end{camlinput}
\end{caml}
\end{camlexample}
\begin{camlexample}{toplevel}
\begin{caml}
\begin{camlinput}
$\?$ dict.Dict.%{"one"};;
\end{camlinput}
\begin{camloutput}
- : int = 1
\end{camloutput}
\end{caml}
\begin{caml}
\begin{camlinput}
$\?$ let open Dict in dict.%{"two"};;
\end{camlinput}
\begin{camloutput}
- : int = 2
\end{camloutput}
\end{caml}
\end{camlexample}

\subsection{ss:multiindexing}{Multi-index notation}
\begin{syntax}
expr:
          ...
        | expr '.' [module-path '.'] dot-ext '(' expr {{';' expr }} ')' [ '<-' expr ]
        | expr '.' [module-path '.'] dot-ext '[' expr {{';' expr }} ']' [ '<-' expr ]
        | expr '.' [module-path '.'] dot-ext '{' expr {{';' expr }} '}' [ '<-' expr ]
;
operator-name:
          ...
        | '.' dot-ext ('(;..)' || '[;..]' || '{;..}') ['<-']
;
\end{syntax}

Multi-index are also supported through a second variant of indexing operators

\begin{camlexample}{verbatim}
\begin{caml}
\begin{camlinput}
let (.%[;..]) = Bigarray.Genarray.get
let (.%{;..}) = Bigarray.Genarray.get
let (.%(;..)) = Bigarray.Genarray.get
\end{camlinput}
\end{caml}
\end{camlexample}

which is called when an index literals contain a semicolon separated list
of expressions with two and more elements:

\begin{camlexample}{verbatim}
\begin{caml}
\begin{camlinput}
let sum x y = x.%[1;2;3] + y.%[1;2]
(* is equivalent to *)
let sum x y = (.%[;..]) x [|1;2;3|] + (.%[;..]) y [|1;2|]
\end{camlinput}
\end{caml}
\end{camlexample}

In particular this multi-index notation makes it possible to uniformly handle
indexing Genarray and other implementations of multidimensional arrays.

\begin{camlexample}{verbatim}
\begin{caml}
\begin{camlinput}
module A = Bigarray.Genarray
let (.%{;..}) = A.get
let (.%{;..}<- ) = A.set
let (.%{ }) a k = A.get a [|k|]
let (.%{ }<-) a k x = A.set a [|k|] x
let syntax_compare vec mat t3 t4 =
          vec.%{0} = A.get vec [|0|]
   &&   mat.%{0;0} = A.get mat [|0;0|]
   &&   t3.%{0;0;0} = A.get t3 [|0;0;0|]
   && t4.%{0;0;0;0} = t4.{0,0,0,0}
\end{camlinput}
\end{caml}
\end{camlexample}

Beware that the differentiation between the multi-index and single index
operators is purely syntactic: multi-index operators are restricted to
index expressions that contain one or more semicolons ";". For instance,
\begin{camlexample}{verbatim}
\begin{caml}
\begin{camlinput}
  let pair vec mat = vec.%{0}, mat.%{0;0}
\end{camlinput}
\end{caml}
\end{camlexample}
is equivalent to
\begin{camlexample}{verbatim}
\begin{caml}
\begin{camlinput}
  let pair vec mat = (.%{ }) vec 0, (.%{;..}) mat [|0;0|]
\end{camlinput}
\end{caml}
\end{camlexample}
Notice that in the "vec" case, we are calling the single index operator, "(.%{})", and
not the multi-index variant, "(.{;..})".
For this reason, it is expected that most users of multi-index operators will need
to define conjointly a single index variant
\begin{camlexample}{verbatim}
\begin{caml}
\begin{camlinput}
let (.%{;..}) = A.get
let (.%{ }) a k = A.get a [|k|]
\end{camlinput}
\end{caml}
\end{camlexample}
to handle both cases uniformly.

\section{s:empty-variants}{Empty variant types}
%HEVEA\cutname{emptyvariants.html}
(Introduced in 4.07.0)

\begin{syntax}
type-representation:
          ...
        | '=' '|'
\end{syntax}
This extension allows user to define empty variants.
Empty variant type can be eliminated by refutation case of pattern matching.
\begin{camlexample}{verbatim}
\begin{caml}
\begin{camlinput}
type t = |
let f (x: t) = match x with _ -> .
\end{camlinput}
\end{caml}
\end{camlexample}

\section{s:alerts}{Alerts}
%HEVEA\cutname{alerts.html}
(Introduced in 4.08)

Since OCaml 4.08, it is possible to mark components (such as value or
type declarations) in signatures with ``alerts'' that will be reported
when those components are referenced.  This generalizes the notion of
``deprecated'' components which were previously reported as warning 3.
Those alerts can be used for instance to report usage of unsafe
features, or of features which are only available on some platforms,
etc.

Alert categories are identified by a symbolic identifier (a lowercase
identifier, following the usual lexical rules) and an optional
message.  The identifier is used to control which alerts are enabled,
and which ones are turned into fatal errors.  The message is reported
to the user when the alert is triggered (i.e. when the marked
component is referenced).

The "ocaml.alert" or "alert" attribute serves two purposes: (i) to
mark component with an alert to be triggered when the component is
referenced, and (ii) to control which alert names are enabled.  In the
first form, the attribute takes an identifier possibly
followed by a message. Here is an example of a value declaration marked
with an alert:

\begin{verbatim}
module U: sig
  val fork: unit -> bool
    [@@alert unix "This function is only available under Unix."]
end
\end{verbatim}

Here "unix" is the identifier for the alert.  If this alert category
is enabled, any reference to "U.fork" will produce a message at
compile time, which can be turned or not into a fatal error.

And here is another example as a floating attribute on top
of an ``.mli'' file (i.e. before any other non-attribute item)
or on top of an ``.ml'' file without a corresponding interface file,
so that any reference to that unit will trigger the alert:

\begin{verbatim}
[@@@alert unsafe "This module is unsafe!"]
\end{verbatim}


Controlling which alerts are enabled and whether they are turned into
fatal errors is done either through the compiler's command-line option
"-alert <spec>" or locally in the code through the "alert" or
"ocaml.alert" attribute taking a single string payload "<spec>".  In
both cases, the syntax for "<spec>" is a concatenation of items of the
form:

\begin{itemize}
\item "+id" enables alert "id".
\item "-id" disables alert "id".
\item "++id" turns alert "id" into a fatal error.
\item "--id" turns alert "id" into non-fatal mode.
\item "\@id" equivalent to "++id+id" (enables "id" and turns it into a fatal-error)
\end{itemize}

As a special case, if "id" is "all", it stands for all alerts.

Here are some examples:

\begin{verbatim}

(* Disable all alerts, reenables just unix (as a soft alert) and window
   (as a fatal-error), for the rest of the current structure *)

[@@@alert "-all--all+unix@window"]
 ...

let x =
  (* Locally disable the window alert *)
  begin[@alert "-window"]
      ...
  end
\end{verbatim}

Before OCaml 4.08, there was support for a single kind of deprecation
alert.  It is now known as the "deprecated" alert, but legacy
attributes to trigger it and the legacy ways to control it as warning
3 are still supported. For instance, passing "-w +3" on the
command-line is equivant to "-alert +deprecated", and:

\begin{verbatim}
val x: int
  [@@@ocaml.deprecated "Please do something else"]
\end{verbatim}

is equivalent to:

\begin{verbatim}
val x: int
  [@@@ocaml.alert deprecated "Please do something else"]
\end{verbatim}

\section{s:generalized-open}{Generalized open statements}
%HEVEA\cutname{generalizedopens.html}

(Introduced in 4.08)

\begin{syntax}
definition:
      ...
   |  'open'  module-expr
   |  'open!' module-expr
;
specification:
      ...
   |  'open'  extended-module-path
   |  'open!' extended-module-path
;
expr:
       ...
     | 'let' 'open'  module-expr 'in' expr
     | 'let' 'open!' module-expr 'in' expr
;
\end{syntax}


This extension makes it possible to open any module expression in
module structures and expressions. A similar mechanism is also available
inside module types, but only for extended module paths (e.g. "F(X).G(Y)").

For instance, a module can be constrained when opened with

\begin{camlexample}{verbatim}
\begin{caml}
\begin{camlinput}
module M = struct let x = 0 let hidden = 1 end
open (M:sig val x: int end)
let y = <<hidden>>
\end{camlinput}
\begin{camlerror}
Error: Unbound value hidden
\end{camlerror}
\end{caml}
\end{camlexample}


Another possibility is to immediately open the result of a functor application

\begin{camlexample}{verbatim}
\begin{caml}
\begin{camlinput}
  let sort (type x) (x:x list) =
    let open Set.Make(struct type t = x let compare=compare end) in
    elements (of_list x)
\end{camlinput}
\begin{camloutput}
val sort : 'x list -> 'x list = <fun>
\end{camloutput}
\end{caml}
\end{camlexample}

Going further, this construction can introduce local components inside a
structure,

\begin{camlexample}{verbatim}
\begin{caml}
\begin{camlinput}
module M = struct
  let x = 0
  open! struct
    let x = 0
    let y = 1
  end
  let w = x + y
end
\end{camlinput}
\begin{camloutput}
module M : sig val x : int val w : int end
\end{camloutput}
\end{caml}
\end{camlexample}

One important restriction is that types introduced by @'open' 'struct' ...
'end'@ cannot appear in the signature of the enclosing structure, unless they
are defined equal to some non-local type.
So:

\begin{camlexample}{verbatim}
\begin{caml}
\begin{camlinput}
module M = struct
  open struct type 'a t = 'a option = None | Some of 'a end
  let x : int t = Some 1
end
\end{camlinput}
\begin{camloutput}
module M : sig val x : int option end
\end{camloutput}
\end{caml}
\end{camlexample}
is OK, but:

\begin{camlexample}{verbatim}
\begin{caml}
\begin{camlinput}
module M = struct
  <<open struct type t = A end>>
  let x = A
end
\end{camlinput}
\begin{camlerror}
Error: The type t/4695 introduced by this open appears in the signature
       File "exten.etex", line 3, characters 6-7:
         The value x has no valid type if t/4695 is hidden
\end{camlerror}
\end{caml}
\end{camlexample}
is not because "x" cannot be given any type other than "t", which only exists
locally. Although the above would be OK if "x" too was local:

\begin{camlexample}{verbatim}
\begin{caml}
\begin{camlinput}
module M: sig end = struct
  open struct
  type t = A
  end
  $\ldots$
  open struct let x = A end
  $\ldots$
end
\end{camlinput}
\begin{camloutput}
module M : sig end
\end{camloutput}
\end{caml}
\end{camlexample}

Inside signatures, extended opens are limited to extended module paths,
\begin{camlexample}{verbatim}
\begin{caml}
\begin{camlinput}
module type S = sig
  module F: sig end -> sig type t end
  module X: sig end
  open F(X)
  val f: t
end
\end{camlinput}
\begin{camloutput}
module type S =
  sig
    module F : sig end -> sig type t end
    module X : sig end
    val f : F(X).t
  end
\end{camloutput}
\end{caml}
\end{camlexample}

and not

\begin{verbatim}
  open struct type t = int end
\end{verbatim}

In those situations, local substitutions(see \ref{ss:local-substitution})
can be used instead.

Beware that this extension is not available inside class definitions:

\begin{verbatim}
class c =
  let open Set.Make(Int) in
  ...
\end{verbatim}

\section{s:binding-operators}{Binding operators}
%HEVEA\cutname{bindingops.html}
(Introduced in 4.08.0)

\begin{syntax}
let-operator:
 | 'let' (core-operator-char || '<') { dot-operator-char }
;
and-operator:
 | 'and' (core-operator-char || '<') { dot-operator-char }
;
operator-name :
          ...
        | let-operator
        | and-operator
;
expr:
          ...
        | let-operator let-binding { and-operator let-binding } in expr
;
\end{syntax}

Users can define {\em let operators}:

\begin{camlexample}{verbatim}
\begin{caml}
\begin{camlinput}
let ( let* ) o f =
  match o with
  | None -> None
  | Some x -> f x

let return x = Some x
\end{camlinput}
\begin{camloutput}
val ( let* ) : 'a option -> ('a -> 'b option) -> 'b option = <fun>
val return : 'a -> 'a option = <fun>
\end{camloutput}
\end{caml}
\end{camlexample}

and then apply them using this convenient syntax:

\begin{camlexample}{verbatim}
\begin{caml}
\begin{camlinput}
let find_and_sum tbl k1 k2 =
  let* x1 = Hashtbl.find_opt tbl k1 in
  let* x2 = Hashtbl.find_opt tbl k2 in
    return (x1 + x2)
\end{camlinput}
\begin{camloutput}
val find_and_sum : ('a, int) Hashtbl.t -> 'a -> 'a -> int option = <fun>
\end{camloutput}
\end{caml}
\end{camlexample}

which is equivalent to this expanded form:

\begin{camlexample}{verbatim}
\begin{caml}
\begin{camlinput}
let find_and_sum tbl k1 k2 =
  ( let* ) (Hashtbl.find_opt tbl k1)
    (fun x1 ->
       ( let* ) (Hashtbl.find_opt tbl k2)
         (fun x2 -> return (x1 + x2)))
\end{camlinput}
\begin{camloutput}
val find_and_sum : ('a, int) Hashtbl.t -> 'a -> 'a -> int option = <fun>
\end{camloutput}
\end{caml}
\end{camlexample}

Users can also define {\em and operators}:

\begin{camlexample}{verbatim}
\begin{caml}
\begin{camlinput}
module ZipSeq = struct

  type 'a t = 'a Seq.t

  open Seq

  let rec return x =
    fun () -> Cons(x, return x)

  let rec prod a b =
    fun () ->
      match a (), b () with
      | Nil, _ | _, Nil -> Nil
      | Cons(x, a), Cons(y, b) -> Cons((x, y), prod a b)

  let ( let+ ) f s = map s f
  let ( and+ ) a b = prod a b

end
\end{camlinput}
\begin{camloutput}
module ZipSeq :
  sig
    type 'a t = 'a Seq.t
    val return : 'a -> 'a Seq.t
    val prod : 'a Seq.t -> 'b Seq.t -> ('a * 'b) Seq.t
    val ( let+ ) : 'a Seq.t -> ('a -> 'b) -> 'b Seq.t
    val ( and+ ) : 'a Seq.t -> 'b Seq.t -> ('a * 'b) Seq.t
  end
\end{camloutput}
\end{caml}
\end{camlexample}

to support the syntax:

\begin{camlexample}{verbatim}
\begin{caml}
\begin{camlinput}
open ZipSeq
let sum3 z1 z2 z3 =
  let+ x1 = z1
  and+ x2 = z2
  and+ x3 = z3 in
    x1 + x2 + x3
\end{camlinput}
\begin{camloutput}
val sum3 : int Seq.t -> int Seq.t -> int Seq.t -> int Seq.t = <fun>
\end{camloutput}
\end{caml}
\end{camlexample}

which is equivalent to this expanded form:

\begin{camlexample}{verbatim}
\begin{caml}
\begin{camlinput}
open ZipSeq
let sum3 z1 z2 z3 =
  ( let+ ) (( and+ ) (( and+ ) z1 z2) z3)
    (fun ((x1, x2), x3) -> x1 + x2 + x3)
\end{camlinput}
\begin{camloutput}
val sum3 : int Seq.t -> int Seq.t -> int Seq.t -> int Seq.t = <fun>
\end{camloutput}
\end{caml}
\end{camlexample}

\subsection{ss:letops-rationale}{Rationale}

This extension is intended to provide a convenient syntax for working
with monads and applicatives.

An applicative should provide a module implementing the following
interface:

\begin{camlexample}{verbatim}
\begin{caml}
\begin{camlinput}
module type Applicative_syntax = sig
  type 'a t
  val ( let+ ) : 'a t -> ('a -> 'b) -> 'b t
  val ( and+ ): 'a t -> 'b t -> ('a * 'b) t
end
\end{camlinput}
\end{caml}
\end{camlexample}

where "(let+)" is bound to the "map" operation and "(and+)" is bound to
the monoidal product operation.

A monad should provide a module implementing the following interface:

\begin{camlexample}{verbatim}
\begin{caml}
\begin{camlinput}
module type Monad_syntax = sig
  include Applicative_syntax
  val ( let* ) : 'a t -> ('a -> 'b t) -> 'b t
  val ( and* ): 'a t -> 'b t -> ('a * 'b) t
end
\end{camlinput}
\end{caml}
\end{camlexample}

where "(let*)" is bound to the "bind" operation, and "(and*)" is also
bound to the monoidal product operation.

%HEVEA\cutend
